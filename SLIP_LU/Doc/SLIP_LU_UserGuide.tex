\documentclass[12pt]{article}
\usepackage{amsmath,amssymb,amsthm,latexsym,paralist,comment}
\usepackage{graphicx}
\usepackage{psfrag}
\usepackage{amsmath}
\usepackage{multirow}
\usepackage{algorithm}
\usepackage{algpseudocode}
\usepackage{cprotect}
\usepackage{graphicx}
\usepackage{subcaption}
\usepackage{listings}
\usepackage[dvipsnames]{xcolor}
\usepackage[hidelinks]{hyperref}
\usepackage{framed}
\usepackage{mdframed}
\hypersetup{
     colorlinks = true,
     citecolor = blue,
     linkcolor = blue,
     urlcolor = Maroon
}
% \usepackage[margin=1in]{geometry}
\usepackage{geometry}
\theoremstyle{definition}
% \setlength{\parskip}{1em}

\newcommand{\N}{\mathbf{N}}
\newcommand{\R}{\mathbf{R}}
\newcommand{\Z}{\mathbf{Z}}
\newcommand{\import}{\textcolor{red}{\textbf{**IMPORTANT**}}}

\begin{document}

\begin{center}
\begin{large}
\textbf{User Guide for SLIP LU, A Sparse Left-Looking Integer
Preserving LU Factorization} \\
\vspace{5mm}
Version 1.0.0, March 2020 % VERSION
\vspace{20mm}

Christopher Lourenco, Jinhao Chen, Erick Moreno-Centeno, Timothy A. Davis \\

Texas A\&M University

\vspace{20mm}
Contact Information: Contact Chris Lourenco, \href{mailto:chrisjlourenco@gmail.com}{chrisjlourenco@gmail.com}, or Tim Davis,
\href{mailto:timdavis@aldenmath.com}{timdavis@aldenmath.com},
\href{mailto:davis@tamu.edu}{davis@tamu.edu},
\href{DrTimothyAldenDavis@gmail.com}{DrTimothyAldenDavis@gmail.com}

\end{large}
\end{center}

\newpage

% Keep table of contents black
{
\small
\hypersetup{ linkcolor = black}
\tableofcontents
}
\newpage

%-------------------------------------------------------------------------------
\section{Summary}
\label{s:intro}
%-------------------------------------------------------------------------------

SLIP LU is a software package designed to exactly solve unsymmetric sparse
linear systems, $ A \mathbf{x} = \mathbf{b}$, where $A \in \mathbb{Q}^{n \times
n}$, $b \in \mathbb{Q}^{n \times m}$, and $\mathbf{x} \in \mathbb{Q}^{n \times
m}$. This package performs a left-looking, roundoff-error-free (REF) LU
factorization $P A Q = L D U$, where $L$ and $U$ are integer, $D$ is diagonal,
and $P$ and $Q$ are row and column permutations, respectively. It is important
to note that the matrix $D$ is never explicitly computed nor needed; thus the
functional form of the factorization requires only the matrices $L$ and $U$.
The theory associated with this code is the Sparse Left-looking
Integer-Preserving (SLIP) LU factorization \cite{lourenco2019exact}. Aside from
solving sparse linear systems exactly, one of the key goals of this package is
to provide a framework for other solvers to benchmark the reliability and
stability of their linear solvers, as our final solution vector $\mathbf{x}$ is
guaranteed to be exact. In addition, SLIP LU provides a wrapper class for the
GNU Multiple Precision Arithmetic (GMP) \cite{granlund2015gnu} and GNU Multiple
Precision Floating Point Reliable (MPFR) \cite{fousse2007mpfr} libraries in
order to prevent memory leaks and improve the overall stability of these
external libraries. SLIP LU is written in ANSI C and is accompanied by a MATLAB
interface.

The user's input matrix $A$ and right hand side (RHS) vectors $\mathbf{b}$ are
read from either \verb|double|, \verb|int|, \verb|mpq_t|, \verb|mpz_t|, or
\verb|mpfr_t| data types. $A$ must be stored in either compressed sparse column
form or sparse triplet form, while $\mathbf{b}$ must be stored as a dense
matrix. A discussion of building each of these types of input is given in
Section \ref{s:Matrix_building_routines}.

The matrices $L$ and $U$ are computed using internal, integer-preserving
routines with the big integer (\verb|mpz_t|) data types from the GMP Library
\cite{granlund2015gnu}. The matrices $L$ and $U$ are computed one column at a
time, where each column is computed via the sparse REF triangular solve
detailed in \cite{lourenco2019exact}. All divisions performed in the algorithm
are guaranteed to be exact (i.e., integer); therefore, no greatest common
divisor algorithms are needed to reduce the size of entries.

The matrices $P$ and $Q$ are either user specified or determined dynamically
during the factorization. For the matrix $P$, the default option is to use a
partial pivoting scheme in which the diagonal entry in column $k$ is selected
if it is the same magnitude as the smallest entry of $k$-th column, otherwise
the smallest entry is selected as the $k$-th pivot. In addition to this
approach, the code allows diagonal pivoting, partial pivoting which selects the
largest pivot, or various tolerance based diagonal pivoting schemes. For the
matrix $Q$, the default ordering is the Column Approximate Minimum Degree
(COLAMD) algorithm \cite{davis2004algorithmcolamd,davis2004column}. Other
approaches include using the Approximate Minimum Degree (AMD) ordering
\cite{amestoy1996approximate,amestoy2004algorithmamd}, a user specified column
ordering (i.e., the default column ordering applied to the input matrix). A
discussion of how to select these matrices prior to factorization is given in
Section \ref{s:UserRoutines}.

Once the factorization $L D U = P A Q $ is computed, the vector $\mathbf{x}$ is
computed via sparse REF forward and backward substitution. The forward
substitution is a variant of the sparse REF triangular solve discussed above.
The backward substitution is a typical column oriented sparse backward
substitution. Both of these routines assume that the right hand side vector(s)
$\mathbf{b}$ are dense. At the conclusion of the forward and backward
substitution routines, the final solution vector(s) $\mathbf{x}$ are guaranteed
to be exact and is stored using the GMP \verb|mpq_t| data structure.

The final phase of SLIP LU comprises output routines. If the user desires it,
their final solution vector(s) can be output in the \verb|mpq_t| data type.
Alternatively, the solution vector(s) can be output in \verb|double| precision
or to any user desired precision via the \verb|mpfr_t| data type. One key
advantage of utilizing SLIP LU with floating-point output is that the solution
is guaranteed to be exact until this final conversion; meaning that roundoff
errors are only introduced in the final conversion from rational numbers. Thus,
the solution vector(s) output in \verb|double| precision are accurate to machine
roundoff (approximately $10^{-16}$) and SLIP LU utilizes higher precision for
the MPFR output; thus it is also accurate to user specified precision.

All left-hand side matrices (referred to as $A$ henceforth) within this package
are stored in compressed sparse column form (CSC). This data structure
stores the matrix $A$ as a sequence of three arrays:

\begin{itemize}
\item
\verb|A->p|: Column pointers; an array of size \verb|n+1|. The row indices of
column $j$ are located in positions \verb|A->p[j]| to \verb|A->p[j+1]-1| of the
array \verb|A->i|. Data type: \verb|int32_t|.

\item
\verb|A->i|: Row indices; an array of size equal to the number of entries in
the matrix. The entry \verb|A->i[k]| is the row index of the $k$th nonzero in
the matrix. Data type: \verb|int32_t|.

\item
\verb|A->x|: Numeric entries. The entry \verb|A->x[k]| is the numeric value of
the $k$th nonzero in the matrix. Data type: \verb|mpz_t|.
\end{itemize}

An example matrix $A$ is stored as follows (notice that via C convention, the
indexing is zero based).
\[
A = \begin{bmatrix}
1 & 0 & 0 & 1 \\
2 & 0 & 4 & 12 \\
7 & 1 & 1 & 1 \\
0 & 2 & 3 & 0 \\
\end{bmatrix}
\]

{\small
\noindent \verb|A->p = [0, 3, 5, 8, 11]| \\
\verb|A->i = [0, 1, 2, 2, 3, 1, 2, 3, 0,  1, 2]| \\
\verb|A->x = [1, 2, 7, 1, 2, 4, 1, 3, 1, 12, 1]|
}

For example, the last column appears in positions 8
to 10 of \verb|A->i| and \verb|A->x|, with row indices 0, 1, and 2, and values
$a_{03}=1$, $a_{13}=12$, and $a_{23}=1$.

%-------------------------------------------------------------------------------
\section{Availability}
%-------------------------------------------------------------------------------

\textbf{Copyright:} This software is copyright by Christopher Lourenco, Jinhao
Chen, Erick Moreno-Centeno, and Timothy Davis.

\noindent \textbf{Contact Info:} Contact Chris Lourenco,
\href{mailto:chrisjlourenco@gmail.com}{chrisjlourenco@gmail.com}, or Tim Davis,
\href{mailto:timdavis@aldenmath.com}{timdavis@aldenmath.com},
\href{mailto:davis@tamu.edu}{davis@tamu.edu}, or
\href{DrTimothyAldenDavis@gmail.com}{DrTimothyAldenDavis@gmail.com}

\noindent \textbf{Licence:} This software package is dual licensed under the
GNU General Public License version 2 or the GNU Lesser General Public License
version 3. Details of this license can be seen in the directory
SLIP\_LU/License/license.txt. In short, SLIP LU is free to use for research
purposes.  For a commercial license, please contact the authors.

\noindent \textbf{Location:} \url{https://github.com/clouren/SLIP_LU} and
\url{www.suitesparse.com}

\noindent \textbf{Required Packages:} SLIP LU requires the installation of AMD
\cite{amestoy1996approximate,amestoy2004algorithmamd}, COLAMD
\cite{davis2004column,davis2004algorithmcolamd}, \verb'SuiteSparse_config'
\cite{davis2020suitesparse}, the GNU GMP \cite{granlund2015gnu} and GNU MPFR
\cite{fousse2007mpfr} libraries.  AMD and COLAMD are available under a BSD
3-clause license, and no license restrictions apply to
\verb'SuiteSparse_config'.  Notice that AMD, COLAMD, and
\verb'SuiteSparse_config' are included in this distribution for users'
convenience. The GNU GMP and GNU MPFR library can be acquired and installed
from \url{https://gmplib.org/} and \url{http://www.mpfr.org/} respectively.

If a user is running Unix that is Debian/Ubuntu based, a compatible version of
GMP and MPFR can be installed with the following terminal commands:

{\small
\begin{verbatim}
    sudo apt-get install libgmp3-dev
    sudo apt-get install libmpfr-dev libmpfr-doc libmpfr4 libmpfr4-dbg
\end{verbatim}
}

%-------------------------------------------------------------------------------
\section{Installation} \label{s:install}
%-------------------------------------------------------------------------------

Installation of SLIP LU requires the \verb|make| utility in Linux or
\verb|Cygwin make| in Windows. With the proper compiler, typing \verb|make|
under the main directory will compile AMD, COLAMD and SLIP LU to the respective
\verb'SLIP_LU/Lib' folder. To further install the libraries onto your computer,
simply type \verb|make install|.  Thereafter, to use the code inside of your
program, precede your code with \verb|#include "SLIP_LU.h"|.

If you want to use SLIP LU within MATLAB, from your installation of MATLAB,
\verb|cd| to the folder \verb|SLIP_LU/SLIP_LU/MATLAB| then type
\verb|SLIP_install|. This should compile the necessary code so that you can use
SLIP LU within MATLAB. Note that this file does not add the correct directory
to your path; therefore, if you want SLIP LU as a default function, type
\verb|pathtool| and save your path for future MATLAB sessions. If you cannot
save your path because of file permissions, edit your \verb|startup.m| by
adding \verb|addpath| commands (type doc startup and doc \verb|addpath| for
more information).

%-------------------------------------------------------------------------------
\section{SLIP LU Data Structures} \label{s:Structures}
%-------------------------------------------------------------------------------

There are four important data structures used throughout the SLIP LU package:
\verb|SLIP_options|, \verb|SLIP_sparse|, \verb|SLIP_dense|, and
\verb|SLIP_LU_analysis|. We describe them briefly below and more in detail in
this section.

\begin{itemize}
    \item \verb|SLIP_options|: Contains numerous command parameters. Default
    values of these parameters are good for a general user; however, modifying
    this struct allows a user to control column orderings, pivoting schemes,
    and other components of the factorization.

    \item \verb|SLIP_sparse|: A sparse matrix for SLIP LU. These matrices are
    stored in the CSC form with \verb|mpz_t| entries.

    \item \verb|SLIP_dense|: A dense matrix for SLIP LU. Primarily used for the
    RHS vector(s) $\mathbf{b}$.

    \item \verb|SLIP_LU_analysis|: A symbolic analysis struct. Contains the
    column permutation and guesses for the number of nonzeros in $L$ and $U$.
\end{itemize}

Furthermore, three enumerated types (\verb|enum|) are defined and used:
\verb|SLIP_pivot|, \verb|SLIP_col_order| and \verb|SLIP_info|. Again we briefly
describe them below and in more detail later in this section.

\begin{itemize}
    \item \verb|SLIP_pivot|: Types of pivoting scheme available for the user.
    \item \verb|SLIP_col_order|: Type of column preordering available for the
    user.
    \item \verb|SLIP_info|: Status codes for SLIP LU. Most function return
    a status indicating success or, in the case of failure, what went wrong.
\end{itemize}

Lastly, SLIP LU defines the following strings with \verb|#define|. Refer to
\verb|SLIP_LU.h| file for details.

%----------------------------------------
% \begin{table*}[htbp]
\begin{center}
\begin{tabular}{ll}
\hline
Macro & purpose \\
\hline
\verb|SLIP_LU_VERSION|       &  current version of the code\\
\verb|SLIP_LU_VERSION_MAJOR| &  major version of the code\\
\verb|SLIP_LU_VERSION_MINOR| & minor version of the code   \\
\verb|SLIP_LU_VERSION_SUB|   &  sub version of the code\\
\hline
\verb|SLIP_PAPER|            & name of associated paper \\
\hline
\verb|SLIP_AUTHOR|           & authors of the code \\
\hline
\end{tabular}
% \label{tab:SLIP_macro}
\end{center}
% \end{table*}

The remainder of this section describes each of these data structures and
enumerated types.

%-------------------------------------------------------------------------------
\cprotect\subsection{\verb|SLIP_info|: status code returned by SLIP LU}
\label{ss:SLIP_info}
%-------------------------------------------------------------------------------

Most of SLIP LU functions return its status to the caller as its return value,
an enumerated type called \verb|SLIP_info|. All possible values for
\verb|SLIP_info| are listed as follows:

% \begin{table*}[htbp]
\begin{center}
\begin{tabular}{rll}
\hline
    0& \verb|SLIP_OK|& The function was successfully executed.\\
\hline
    -1& \verb|SLIP_OUT_OF_MEMORY|& out of memory\\
\hline
    -2& \verb|SLIP_SINGULAR|& The input matrix $A$ is exactly singular.\\
\hline
    -3& \verb|SLIP_INCORRECT_INPUT|& One or more input arguments are incorrect.\\
\hline
    -4& \verb|SLIP_INCORRECT|& The solution is incorrect.\\
\hline
\end{tabular}
% \label{tab:SLIP_info}
\end{center}
% \end{table*}

%-------------------------------------------------------------------------------
\cprotect\subsection{\verb|SLIP_pivot|: enum for pivoting schemes}
\label{ss:SLIP_pivot}
%-------------------------------------------------------------------------------

There are six available pivoting schemes provided in SLIP LU.  Users can set
the pivoting method through the \verb|SLIP_options| structure in Section
\ref{ss:SLIP_options}. Note that the pivot is always nonzero, thus the smallest
entry is the nonzero entry with the smallest magnitude.  Also, the tolerance is
specified by the \verb|tol| component in \verb|SLIP_options|.  Please refer to
Section \ref{ss:SLIP_options} for details of this parameter. The pivoting
schemes are described as follows:

%----------------------------------------
{\small
\begin{center}
\begin{tabular}{llp{4in}}
\hline
0 & \verb|SLIP_SMALLEST|        & The $k$-th pivot is selected as the smallest
                                  entry in the $k$th column.\\
\hline
1 & \verb|SLIP_DIAGONAL|        & The $k$-th pivot is selected as the diagonal
                                  entry. If the diagonal entry is zero,
                                  this method instead selects the smallest
                                  pivot in the column.\\
\hline
2 & \verb|SLIP_FIRST_NONZERO|   & The $k$-th pivot is selected as the first
                                  eligible nonzero in the column. \\
\hline
3 & \verb|SLIP_TOL_SMALLEST|    & The $k$-th pivot is selected as the diagonal
                                  entry if the diagonal is within a
                                  specified tolerance of the smallest entry in
                                  the column. Otherwise, the smallest
                                  entry in the $k$-th column is selected.
                                  This is the default pivot selection
                                  strategy. \\
\hline
4 & \verb|SLIP_TOL_LARGEST|     & The $k$-th pivot is selected as the diagonal
                                  entry if the diagonal is within a
                                  specified tolerance of the largest entry in
                                  the column.  Otherwise, the largest
                                  entry in the $k$-th column is selected. \\
\hline
5 & \verb|SLIP_LARGEST|         & The $k$-th pivot is selected as the largest
                                  entry in the $k$-th column. \\
\hline
\end{tabular}
\end{center}
}

%-------------------------------------------------------------------------------
\cprotect\subsection{\verb|SLIP_col_order|: enum for column ordering schemes}
\label{ss:SLIP_col_order}
%-------------------------------------------------------------------------------

The SLIP LU library provides three column ordering schemes: no ordering,
COLAMD, and AMD. Users can set the column ordering method through \verb|order|
component in the \verb|SLIP_option| structure described in Section
\ref{ss:SLIP_options}. In general, it is recommended that the user selects the
COLAMD ordering, however, no preordering can be preferable if the user's matrix
already has a good preordering.

{\small
% \begin{table*}[htbp]
\begin{center}
\begin{tabular}{llp{4in}}
\hline
0 & \verb|SLIP_NO_ORDERING| & No pre-ordering is performed on the matrix $A$,
                              that is $Q = I$. \\
\hline
1 & \verb|SLIP_COLAMD|      & The columns of $A$ are permuted prior to
                              factorization using the COLAMD
                              \cite{davis2004algorithmcolamd} ordering.
                              This is the default ordering. \\
\hline
2 & \verb|SLIP_AMD|         & The nonzero pattern of $A + A^T$ is analyzed and
                              the columns of $A$ are permuted prior to
                              factorization based on the AMD
                              \cite{amestoy2004algorithmamd} ordering of
                              $A+A^T$. This works well if $A$ has a mostly
                              symmetric pattern, but tends to be worse
                              than COLAMD on matrices with unsymmetric pattern.
                              \cite{davis2004column}.\\
\hline
\end{tabular}
\label{tab:SLIP_pivot}
\end{center}
% \end{table*}
}

%-------------------------------------------------------------------------------
\cprotect\subsection{ \verb|SLIP_options| structure}
\label{ss:SLIP_options}
%-------------------------------------------------------------------------------

The \verb|SLIP_options| struct stores key command parameters for various
functions used in the SLIP LU package. The \verb|SLIP_options* option| struct
contains the following components:

\begin{itemize}
\item
\verb|option->pivot|: An enum \verb|SLIP_pivot| type (discussed in Section
\ref{ss:SLIP_pivot}) which controls the type of pivoting used. Default value:
\verb|SLIP_TOL_SMALLEST| (3).

\item
\verb|option->order|: An enum \verb|SLIP_col_order| type (discussed in Section
\ref{ss:SLIP_col_order}) which controls what column ordering is used. Default
value: \verb|SLIP_COLAMD| (1).

\item
\verb|option->tol|: A \verb|double| which tells the tolerance used if the user
selects a tolerance based pivoting scheme, i.e., \verb|SLIP_TOL_SMALLEST| or
\verb|SLIP_TOL_LARGEST|. \verb|option->tol| must be in the range of $(0,1]$.
Default value: 1 meaning that the diagonal entry will be selected if it has the
same magnitude as the smallest entry in the $k$ the column.

\item
\verb|option->print_level|: An \verb|int32_t| which controls the amount of
output. 0: print nothing, 1: just errors, 2: terse, with basic stats from
COLAMD/AMD and SLIP, 3: all, with matrices and results. Default value: 0.

\item
\verb|option->prec|: An \verb|uint64_t| which specifies the precision used if
the user desires multiple precision floating point numbers, (i.e., MPFR). This
can be any integer larger than \verb|MPFR_PREC_MIN| (value of 1 in MPFR 4.0.2
and 2 in some legacy versions) and smaller than \verb|MPFR_PREC_MAX| (usually
the largest possible \verb'int' available in your system). Default value: 128
(quad precision).

\item
\verb|option->SLIP_MPFR_ROUND|: A \verb|mpfr_rnd_t| which determines the type
of MPFR rounding to be used by SLIP LU. This is a parameter of the MPFR
library. The options for this parameter are:

    \begin{itemize}
        \item \verb|MPFR_RNDN|: round to nearest
            (roundTiesToEven in IEEE 754-2008)
        \item \verb|MPFR_RNDZ|: round toward zero
            (roundTowardZero in IEEE 754-2008)
        \item \verb|MPFR_RNDU|: round toward plus infinity
            (roundTowardPositive in IEEE 754-2008)
        \item \verb|MPFR_RNDD|: round toward minus infinity
            (roundTowardNegative in IEEE 754-2008)
        \item \verb|MPFR_RNDA|: round away from zero
        \item \verb|MPFR_RNDF|: faithful rounding. This is not stable.
    \end{itemize}

\noindent By default, SLIP LU utilizes \verb|MPFR_RNDN|. We refer the reader to
the MPFR user guide available at
\url{https://www.mpfr.org/mpfr-current/mpfr.pdf} for details on the MPFR
rounding style and any other utilized MPFR convention.

\end{itemize}

The SLIP LU package uses the following function/macro to create and destroy a
\verb|SLIP_options| object.

%----------------------------------------
\begin{center}
\begin{tabular}{lp{2.5in}l}
\hline
function/macro name & description & section \\
\hline
\verb|SLIP_create_default_options|
    & create and return \verb|SLIP_options| pointer
      with default parameters upon successful allocation
    & \ref{ss:create_default_options} \\
\hline
\verb|SLIP_FREE|
    & destroy \verb|SLIP_options| object
    & \ref{ss:SLIP_free} \\
\hline
\end{tabular}
\end{center}

%-------------------------------------------------------------------------------
\cprotect\subsection{The \verb|SLIP_sparse| structure}
\label{ss:SLIP_sparse}
%-------------------------------------------------------------------------------

All internal sparse matrices are stored in compressed sparse column (CSC)
form via the \verb|SLIP_sparse| structure. A sparse matrix
\verb|SLIP_sparse *A| has the following components:

\begin{itemize}
\item \verb|A->m|: Number of rows in the matrix. It is typically assumed that
$m=n$.  Data Type: \verb|int32_t|

\item \verb|A->n|: Number of columns in the matrix. It is typically assumed
that $m=n$. Data Type: \verb|int32_t|

\item \verb|A->nz|: The number of nonzeros in the matrix $A$. Data Type:
\verb|int32_t|

\item \verb|A->nzmax|: The allocated size of the vectors \verb|A->x| and
\verb|A->i|. Note that \verb|A->nzmax| $\geq$ \verb|A->nz|. Internally, this
parameter serves as an estimate on the amount of memory needed and is used to
reduce the number of intermediate reallocations performed in the library. Data
Type: \verb|int32_t|

\item \verb|A->p|: An array of size $n+1$ which contains column pointers of
$A$. Data Type: \verb|int32_t*|

\item \verb|A->i|: An array of size \verb|A->nzmax| which contains the row
indices of the nonzeros in $A$. The matrix is zero based therefore indices are
in the range of $[0, n-1]$. Data Type: \verb|int32_t*|

\item \verb|A->x|: An array of size \verb|A->nzmax| which contains the numeric
values of the matrix. Data Type: \verb|mpz_t*|

\item \verb|A->scale|: A scaling parameter that ensures integrality if the
input sparse matrix is stored as either double, variable precision floating
point, or rational. Data Type: \verb|mpq_t|

\end{itemize}

The SLIP LU package has a set of functions to create, build and destroy a SLIP
LU sparse matrix, \verb|SLIP_sparse|, as shown in the following table.

%----------------------------------------
{\small
 %\begin{table*}[htbp]
\begin{center}
\begin{tabular}{lp{2.5in}l}
\hline
function name & description & section \\
\hline
\verb|SLIP_build_sparse_csc_double|
    & build sparse matrix from \verb|double| type CSC matrix
    & \ref{s:user:build_sparse_csc_double} \\
\verb|SLIP_build_sparse_csc_int|
    & build sparse matrix from \verb|int32_t| type CSC matrix
    & \ref{s:user:build_sparse_csc_int} \\
\verb|SLIP_build_sparse_csc_mpq|
    & build sparse matrix from \verb|mpq_t| type CSC matrix
    & \ref{s:user:build_sparse_csc_mpq} \\
\verb|SLIP_build_sparse_csc_mpfr|
    & build sparse matrix from \verb|mpfr_t| type CSC matrix
    & \ref{s:user:build_sparse_csc_mpfr} \\
\verb|SLIP_build_sparse_csc_mpz|
    & build sparse matrix from \verb|mpz_t| type CSC matrix
    & \ref{s:user:build_sparse_csc_mpz} \\
\hline
\verb|SLIP_build_sparse_trip_double|
    & build sparse matrix from \verb|double| type triplet-format matrix
    & \ref{s:user:build_sparse_trip_double} \\
\verb|SLIP_build_sparse_trip_int|
    & build sparse matrix from \verb|int32_t| type triplet-format matrix
    & \ref{s:user:build_sparse_trip_int} \\
\verb|SLIP_build_sparse_trip_mpq|
    & build sparse matrix from \verb|mpq_t| type triplet-format matrix
    & \ref{s:user:build_sparse_trip_mpq} \\
\verb|SLIP_build_sparse_trip_mpfr|
    & build sparse matrix from \verb|mpfr_t| type triplet-format matrix
    & \ref{s:user:build_sparse_trip_mpfr} \\
\verb|SLIP_build_sparse_trip_mpz|
    & build sparse matrix from \verb|mpz_t| type triplet-format matrix
    & \ref{s:user:build_sparse_trip_mpz} \\
\hline
\verb|SLIP_delete_sparse|
    & destroy sparse matrix
    & \ref{ss:delete_sparse}\\
\hline
\end{tabular}
\end{center}
% \end{table*}
}


%-------------------------------------------------------------------------------
\cprotect\subsection{The \verb|SLIP_dense| structure}
\label{ss:SLIP_dense}
%-------------------------------------------------------------------------------

All internal right-hand side matrices are stored as dense matrices, using the
\verb|SLIP_dense| structure. A dense matrix \verb|SLIP_dense *b| has the
following components:

\begin{itemize}
\item \verb|b->m|: Number of rows in the matrix. Data Type: \verb|int32_t|
\item \verb|b->n|: Number of columns in the matrix. Data Type: \verb|int32_t|
\item \verb|b->x|: A 2D array of size $m$-by-$n$ which contains the numeric
    values of the matrix. Data Type: \verb|mpz_t**|
\item \verb|b->scale|: A scaling parameters that ensures integrality if the
    input dense matrix is stored as either double, variable precision floating
    point, or rational. Data Type: \verb|mpq_t|
\end{itemize}

The SLIP LU package has a set of functions to create, build and destroy a SLIP
LU dense matrix, \verb|SLIP_dense|, described in the following table:

%----------------------------------------
{\small
% \begin{table*}[htbp]
\begin{center}
\begin{tabular}{lll}
\hline
function name & description & section \\
\hline
\verb|SLIP_build_dense_double|
    & build dense matrix from 2D \verb|double| array
    & \ref{s:user:build_dense_double} \\
\verb|SLIP_build_dense_int|
    & build dense matrix from 2D \verb|int32_t| array
    & \ref{s:user:build_dense_int} \\
\verb|SLIP_build_dense_mpq|
    & build dense matrix from 2D \verb|mpq_t| array
    & \ref{s:user:build_dense_mpq} \\
\verb|SLIP_build_dense_mpfr|
    & build dense matrix from 2D \verb|mpfr_t| array
    & \ref{s:user:build_dense_mpfr} \\
\verb|SLIP_build_dense_mpz|
    & build dense matrix from 2D \verb|mpz_t| array
    & \ref{s:user:build_dense_mpz} \\
\hline
\verb|SLIP_delete_dense|
    & destroy dense matrix
    & \ref{ss:delete_dense}\\
\hline
\end{tabular}
\end{center}
% \end{table*}
}

%-------------------------------------------------------------------------------
\cprotect\subsection{\verb|SLIP_LU_analysis| structure}
\label{ss:SLIP_LU_analysis}
%-------------------------------------------------------------------------------

The \verb|SLIP_LU_analysis| data structure is used for storing the column
permutation for LU and the guess on nonzeros for $L$ and $U$. Users do not need
to modify this struct, just pass it into the functions. A
\verb|SLIP_LU_analysis| structure has the following components:

\begin{itemize}
\item \verb|S->q|: The column permutation stored as a dense \verb|int32_t|
vector of size $n+1$, where $n$ is the number of columns of the analyzed matrix.
Currently this vector is obtained via COLAMD, AMD, or is set to no ordering
(i.e., $[0, 1, \hdots, n-1]$).

\item \verb|S->lnz|: An \verb|int32_t| which is a guess for the number of
nonzeros in $L$. \verb|S->lnz| must be in the range of $[n, n^2]$. If
\verb|S->lnz| is too small, the program may waste time performing extra memory
reallocations. This is set during the symbolic analysis.

\item \verb|S->unz|: An \verb|int32_t| which is a guess for the number of
nonzeros in $U$. \verb|S->unz| must be in the range of $[n, n^2$. If
\verb|S->unz| is too small, the program may waste time performing extra memory
reallocations. This is set during the symbolic analysis.
\end{itemize}

The SLIP LU package provides the following functions to create and destroy a
\verb|SLIP_LU_analysis| object:

%----------------------------------------
{\small
% \begin{table*}[htbp]
\begin{center}
\begin{tabular}{lll}
\hline
function/macro name & description & section \\
\hline
\verb|SLIP_LU_analyze|
    & create \verb|SLIP_LU_analysis| object
    & \ref{s:SLIP_LU_analyze} \\
\hline
\verb|SLIP_delete_LU_analysis|
    & destroy \verb|SLIP_LU_analysis| object
    & \ref{ss:delete_LU_analysis} \\
\hline
\end{tabular}
\end{center}
% \end{table*}
}

%-------------------------------------------------------------------------------
\cprotect\section{SLIP LU User-Callable Routines}
\label{s:UserRoutines}
%-------------------------------------------------------------------------------

This section descibes all user callable routines in SLIP LU. A comprehensive
list of these routines also appears in the \verb|SLIP_LU.h| file. For each of
these functions, this section describes its purpose, syntax, input arguments,
and output.

%-------------------------------------------------------------------------------
\subsection{Memory Management Routines} \label{s:user:memmanag}
%-------------------------------------------------------------------------------

The routines in this section are used to allocate and free memory for the data
structures used in SLIP LU.

%-------------------------------------------------------------------------------
\cprotect\subsubsection{\verb|SLIP_calloc|: allocate initialized memory}
\label{ss:SLIP_calloc}
%-------------------------------------------------------------------------------

\begin{mdframed}[userdefinedwidth=6in]
{\footnotesize
\begin{verbatim}
    void *SLIP_calloc
    (
        size_t n,          // Size of array
        size_t size        // Size to alloc
    );
\end{verbatim}
} \end{mdframed}

\verb|SLIP_calloc| allocates a block of memory for an array of \verb|n|
elements, each of them \verb|size| bytes long, and initializes all its bits to
zero.  If any input is equal to zero, it is treated as if equal to 1.  If the
function failed to allocate the requested block of memory, then a \verb|NULL|
pointer is returned.

%-------------------------------------------------------------------------------
\cprotect\subsubsection{\verb|SLIP_malloc|: allocate uninitialized memory}
\label{ss:SLIP_malloc}
%-------------------------------------------------------------------------------

\begin{mdframed}[userdefinedwidth=6in]
{\footnotesize
\begin{verbatim}
    void * SLIP_malloc
    (
        size_t size        // Size to alloc
    );
\end{verbatim}
} \end{mdframed}

\verb|SLIP_malloc| allocates a block of \verb|size| bytes of memory, returning
a pointer to the beginning of the block. The content of the newly allocated
block of memory is not initialized, remaining with indeterminate values.  If
the \verb|size| is zero, it is treated as if equal to 1.  If the function fails
to allocate the requested block of memory, then a \verb|NULL| pointer is
returned.

%-------------------------------------------------------------------------------
\cprotect\subsubsection{\verb|SLIP_realloc|: resize allocated memory}
\label{ss:SLIP_realloc}
%-------------------------------------------------------------------------------

\begin{mdframed}[userdefinedwidth=6in]
{\footnotesize
\begin{verbatim}
    void* SLIP_realloc
    (
        void *p,            // Pointer to array to be realloced
        size_t old_size,    // Old size of the array
        size_t new_size     // New size of the array
    );
\end{verbatim}
} \end{mdframed}

\verb|SLIP_realloc| attempts to resize the memory block pointed to by \verb|p|
that was previously allocated with a call to \verb|SLIP_malloc| or
\verb|SLIP_calloc|. In the case when the function fails to allocate new block
of memory as required and the newly required memory size is smaller than the
old one, then the old block is kept unchanged and \verb|SLIP_realloc| pretends
to succeed. Otherwise, the function returns either \verb|NULL| when it fails,
or the new block of memory when it succeeds.

%-------------------------------------------------------------------------------
\cprotect\subsubsection{\verb|SLIP_free|: free allocated memory}
\label{ss:SLIP_free}
%-------------------------------------------------------------------------------

\begin{mdframed}[userdefinedwidth=6in]
{\footnotesize
\begin{verbatim}
    void SLIP_free
    (
        void *p         // Pointer to be free'd
    );
\end{verbatim}
} \end{mdframed}

\verb|SLIP_free| deallocates the memory previously allocated by a call to
\verb|SLIP_calloc|, \verb|SLIP_malloc|, or \verb|SLIP_realloc|. Note that the
default C \verb|free| function can cause a segmentation fault if called
multiple times on the same pointer or is called via other inappropriate
behavior. To remedy this issue, this function frees the input pointer \verb|p|
only when it is not \verb|NULL|. To further prevent the potential segmentation
fault that could be caused by \verb|free|, the following macro \verb|SLIP_FREE|
is provided, which sets the free'd pointer to \verb|NULL|.

\begin{mdframed}[userdefinedwidth=6in]
{\footnotesize
\begin{verbatim}
    #define SLIP_FREE(p)                        \
    {                                           \
        SLIP_free (p) ;                         \
        (p) = NULL ;                            \
    }
\end{verbatim}
} \end{mdframed}

%-------------------------------------------------------------------------------
\cprotect\subsubsection{\verb|SLIP_initialize|: initialize the working environment}
%-------------------------------------------------------------------------------

\begin{mdframed}[userdefinedwidth=6in]
{\footnotesize
\begin{verbatim}
    void SLIP_initialize
    (
        void
    );
\end{verbatim}
} \end{mdframed}

\verb|SLIP_initialize| initializes the working environment for SLIP LU
functions. SLIP LU utilizes a specialized memory management scheme in order to
prevent potential memory failures caused by GMP library. This function
\textcolor{blue}{\textbf{must}} be called prior to using the library. See the
next section \verb|SLIP_initialize_expert| for more details.

%-------------------------------------------------------------------------------
\cprotect\subsubsection{\verb|SLIP_initialize_expert|: initialize the working environment (expert version)}\label{ss:SLIP_initialize_expert}
%-------------------------------------------------------------------------------

\begin{mdframed}[userdefinedwidth=6in]
{\footnotesize
\begin{verbatim}
    void SLIP_initialize_expert
    (
        void* (*MyMalloc) (size_t),                 // User defined malloc
        void* (*MyRealloc) (void *, size_t, size_t),// User defined realloc
        void (*MyFree) (void*, size_t)              // User defined free
    );
\end{verbatim}
} \end{mdframed}

\verb|SLIP_initialize_expert| initializes the working environment for SLIP LU
with custom memory functions that are used for GMP. If the user passes in their
own \verb|malloc|, \verb|realloc|, or \verb|free| function(s), we use those
internally to process memory. If a NULL pointer is passed in for any function,
then default functions are used.

The three functions are similar to ANSI C \verb|malloc|, \verb|realloc|, and
\verb|free| functions, but the calling syntax is not the same.  Below are the
definitions that \textcolor{blue}{\textbf{must}} be followed, per the GMP
specification:

\begin{mdframed}[userdefinedwidth=6in]
{\footnotesize
\begin{verbatim}
    void *MyMalloc (size_t size) ;  // same as the ANSI C malloc
    void *MyRealloc (void *p, size_t oldsize, size_t newsize) ; // differs
    void MyFree (void *p, size_t size) ; // differs
\end{verbatim}
} \end{mdframed}

\verb|MyMalloc| has identical parameters as the the ANSI C \verb|malloc|.
\verb|MyRealloc| adds a parameter, \verb|oldsize|, which is the prior size of
the block of memory to be reallocated.  \verb|MyFree| takes a second argument,
which is the size of the block that is being free'd.

The default memory management functions used inside of SLIP LU's GMP interface
are:

\begin{mdframed}[userdefinedwidth=6in]
{\footnotesize
\begin{verbatim}
    MyMalloc    slip_gmp_allocate
    MyRealloc   slip_gmp_reallocate
    MyFree      slip_gmp_free
\end{verbatim}
} \end{mdframed}

The \verb|slip_gmp_*| memory management functions are unique to SLIP LU
Library.  They provide an elegant workaround for how GMP manages its memory.
By default, if GMP attempts to allocate memory, but it fails, then it simply
terminates the user application.  This behavior is not suitable for many
applications (MATLAB in particular).  Fortunately, GMP allows the user
application (SLIP LU in this case) to pass in alternative memory manager
functions, via \verb|mp_set_memory_functions|.  The \verb|slip_gmp_*| functions
do not return to GMP if the allocation fails, but instead use the
\verb|longjmp| feature of ANSI C to implement a try/catch mechanism.  The
memory failure can then be safely handled by SLIP LU, without memory leaks and
without terminating the user application.

When SLIP LU is used via MATLAB, the following functions are used instead:

\begin{mdframed}[userdefinedwidth=6in]
{\footnotesize
\begin{verbatim}
    MyMalloc    mxMalloc
    MyRealloc   slip_gmp_mex_realloc (a wrapper for mxRealloc)
    MyFree      slip_gmp_mex_free (a wrapper for mxFree)
\end{verbatim}
} \end{mdframed}

Note that these functions are not used by SLIP LU itself, but only inside GMP.
The functions used by SLIP LU itself are \verb|SLIP_malloc|,
\verb|SLIP_calloc|, \verb|SLIP_realloc|, and \verb|SLIP_free|, which are
wrappers for the ANSI C \verb|malloc|, \verb|calloc|, \verb|realloc|, and
\verb|free| (see Sections \ref{ss:SLIP_calloc}-\ref{ss:SLIP_free}), or (if used
inside MATLAB), for the MATLAB \verb|mxMalloc|, \verb|mxCalloc|,
\verb|mxRealloc|, and \verb|mxFree| functions.

%-------------------------------------------------------------------------------
\cprotect\subsubsection{\verb|SLIP_finalize|: free the working environment}
\label{ss:SLIP_finalize}
%-------------------------------------------------------------------------------

\begin{mdframed}[userdefinedwidth=6in]
{\footnotesize
\begin{verbatim}
    void SLIP_finalize
    (
        void
    );
\end{verbatim}
} \end{mdframed}

\verb|SLIP_finalize| frees the working environment for SLIP LU library. SLIP LU
utilizes a specialized memory management scheme in order to prevent memory
failures. Calling the function \verb|SLIP_finalize| after you are finished
using the library ensures all memory is freed.

%-------------------------------------------------------------------------------
\cprotect\subsubsection{\verb|SLIP_create_default_options|: create default \verb|SLIP_option| object}
\label{ss:create_default_options}
%-------------------------------------------------------------------------------

\begin{mdframed}[userdefinedwidth=6in]
{\footnotesize
\begin{verbatim}
    SLIP_options* SLIP_create_default_options
    (
        void
    );
\end{verbatim}
} \end{mdframed}

\verb|SLIP_create_default_options| creates and returns a pointer to a
\verb|SLIP_options| struct with default parameters upon successful allocation,
which are discussed in Section \ref{ss:SLIP_options}.  To safely free the
\verb|SLIP_options* option| structure, simply use \verb|SLIP_FREE(option)|.

%-------------------------------------------------------------------------------
\cprotect\subsubsection{\verb|SLIP_delete_sparse|: delete sparse matrix}
\label{ss:delete_sparse}
%-------------------------------------------------------------------------------

\begin{mdframed}[userdefinedwidth=6in]
{\footnotesize
\begin{verbatim}
    void SLIP_delete_sparse
    (
        SLIP_sparse **A // matrix to be deleted
    );
\end{verbatim}
} \end{mdframed}

\verb|SLIP_delete_sparse| deletes the sparse matrix \verb|A|,
which is then set to \verb|NULL|.

%-------------------------------------------------------------------------------
\cprotect\subsubsection{\verb|SLIP_delete_dense|: delete dense matrix}
\label{ss:delete_dense}
%-------------------------------------------------------------------------------

\begin{mdframed}[userdefinedwidth=6in]
{\footnotesize
\begin{verbatim}
    void SLIP_delete_dense
    (
        SLIP_dense **A
    );
\end{verbatim}
} \end{mdframed}

\verb|SLIP_delete_dense| deletes the dense matrix \verb|A|,
which is then set to \verb|NULL|.

%-------------------------------------------------------------------------------
\cprotect\subsubsection{\verb|SLIP_delete_LU_analysis|: delete \verb|SLIP_LU_analysis| structure}
\label{ss:delete_LU_analysis}
%-------------------------------------------------------------------------------

\begin{mdframed}[userdefinedwidth=6in]
{\footnotesize
\begin{verbatim}
    void SLIP_delete_LU_analysis
    (
        SLIP_LU_analysis **S // Structure to be deleted
    );
\end{verbatim}
} \end{mdframed}


\verb|SLIP_delete_LU_analysis| deletes a \verb|SLIP_LU_analysis| structure.
Note that the input of the function is the pointer to the pointer of a
\verb|SLIP_LU_analysis| structure. This is because this function internally
sets the pointer of a \verb|SLIP_LU_analysis| to be \verb|NULL| to prevent
potential segmentation fault that could be caused by double \verb|free|.

%-------------------------------------------------------------------------------
\subsection{Matrix Building Routines} \label{s:Matrix_building_routines}
%-------------------------------------------------------------------------------

The routines in this section are used to build either the sparse matrix or the
dense matrix.

%-------------------------------------------------------------------------------
\cprotect\subsubsection{\verb|SLIP_build_sparse_csc_double|: build sparse matrix using CSC with \verb|double| entries}
\label{s:user:build_sparse_csc_double}
%-------------------------------------------------------------------------------

\begin{mdframed}[userdefinedwidth=6in]
{\footnotesize
\begin{verbatim}
    SLIP_info SLIP_build_sparse_csc_double
    (
        SLIP_sparse **A_handle,     // output matrix
        int32_t *p,         // The set of column pointers
        int32_t *I,         // set of row indices
        double *x,          // Set of values as doubles
        int32_t n,          // dimension of the matrix
        int32_t nz          // number of nonzeros in A (size of x and I vectors)
    );
\end{verbatim}
} \end{mdframed}

\verb|SLIP_build_sparse_csc_double| builds a sparse matrix using compressed
sparse column (CSC) form inputs, where the entry values are \verb|double| type.

%-------------------------------------------------------------------------------
\cprotect\subsubsection{\verb|SLIP_build_sparse_csc_int|: build sparse matrix using CSC with \verb|int32_t| entries}
\label{s:user:build_sparse_csc_int}
%-------------------------------------------------------------------------------

\begin{mdframed}[userdefinedwidth=6in]
{\footnotesize
\begin{verbatim}
    SLIP_info SLIP_build_sparse_csc_int
    (
        SLIP_sparse **A_handle,     // output matrix
        int32_t *p,         // The set of column pointers
        int32_t *I,         // set of row indices
        int32_t *x,         // Set of values as doubles
        int32_t n,          // dimension of the matrix
        int32_t nz          // number of nonzeros in A (size of x and I vectors)
    );
\end{verbatim}
} \end{mdframed}

\verb|SLIP_build_sparse_csc_int| builds a sparse matrix using compressed column
form inputs, where the entry values are \verb|int32_t| type.

%-------------------------------------------------------------------------------
\cprotect\subsubsection{\verb|SLIP_build_sparse_csc_mpq|: build sparse matrix using CSC with \verb|mpq_t| entries}
\label{s:user:build_sparse_csc_mpq}
%-------------------------------------------------------------------------------

\begin{mdframed}[userdefinedwidth=6in]
{\footnotesize
\begin{verbatim}
    SLIP_info SLIP_build_sparse_csc_mpq
    (
        SLIP_sparse **A_handle,     // output matrix
        int32_t *p,         // The set of column pointers
        int32_t *I,         // set of row indices
        mpq_t *x,           // Set of values as mpq_t rational numbers
        int32_t n,          // dimension of the matrix
        int32_t nz          // number of nonzeros in A (size of x and I vectors)
    );
\end{verbatim}
} \end{mdframed}

\verb|SLIP_build_sparse_csc_mpq| builds a sparse matrix using compressed
sparse column (CSC) form inputs, where the entry values are \verb|mpq_t| type.

%-------------------------------------------------------------------------------
\cprotect\subsubsection{\verb|SLIP_build_sparse_csc_mpfr|: build sparse matrix using CSC with \verb|mpfr_t| entries}
\label{s:user:build_sparse_csc_mpfr}
%-------------------------------------------------------------------------------

\begin{mdframed}[userdefinedwidth=6in]
{\footnotesize
\begin{verbatim}
    SLIP_info SLIP_build_sparse_csc_mpfr
    (
        SLIP_sparse **A_handle,     // output matrix
        int32_t *p,         // The set of column pointers
        int32_t *I,         // set of row indices
        mpfr_t *x,          // Set of values as doubles
        int32_t n,          // dimension of the matrix
        int32_t nz,         // number of nonzeros in A (size of x and I vectors)
        SLIP_options *option  // command options containing the prec for mpfr
    );
\end{verbatim}
} \end{mdframed}

\verb|SLIP_build_sparse_csc_mpfr| builds a sparse matrix using compressed
sparse column (CSC) form inputs, where the entry values are \verb|mpfr_t| type.

%-------------------------------------------------------------------------------
\cprotect\subsubsection{\verb|SLIP_build_sparse_csc_mpz|: build sparse matrix using CSC with \verb|mpz_t| entries}
\label{s:user:build_sparse_csc_mpz}
%-------------------------------------------------------------------------------

\begin{mdframed}[userdefinedwidth=6in]
{\footnotesize
\begin{verbatim}
    SLIP_info SLIP_build_sparse_csc_mpz
    (
        SLIP_sparse **A_handle,     // output matrix
        int32_t *p,         // The set of column pointers
        int32_t *I,         // set of row indices
        mpz_t *x,           // Set of values in full precision int.
        int32_t n,          // dimension of the matrix
        int32_t nz          // number of nonzeros in A (size of x and I vectors)
    );
\end{verbatim}
} \end{mdframed}

\verb|SLIP_build_sparse_csc_mpz| builds a sparse matrix using compressed column
form inputs, where the entry values are \verb|mpz_t| type.

%-------------------------------------------------------------------------------
\cprotect\subsubsection{\verb|SLIP_build_sparse_trip_double|: build sparse matrix using triplet with \verb|double| entries}
\label{s:user:build_sparse_trip_double}
%-------------------------------------------------------------------------------

\begin{mdframed}[userdefinedwidth=6in]
{\footnotesize
\begin{verbatim}
    SLIP_info SLIP_build_sparse_trip_double
    (
        SLIP_sparse **A_handle,     // output matrix
        int32_t *I,         // set of row indices
        int32_t *J,         // set of column indices
        double *x,          // Set of values in double
        int32_t n,          // dimension of the matrix
        int32_t nz          // number of nonzeros in A (size of x, I,
                            // and J vectors)
    );
\end{verbatim}
} \end{mdframed}

\verb|SLIP_build_sparse_trip_double| builds a sparse matrix using triplet form
inputs, where the entry values are \verb|double| type.

%-------------------------------------------------------------------------------
\cprotect\subsubsection{\verb|SLIP_build_sparse_trip_int|: build sparse matrix using triplet with \verb|int32_t| entries}
\label{s:user:build_sparse_trip_int}
%-------------------------------------------------------------------------------

\begin{mdframed}[userdefinedwidth=6in]
{\footnotesize
\begin{verbatim}
    SLIP_info SLIP_build_sparse_trip_int
    (
        SLIP_sparse **A_handle,     // output matrix
        int32_t *I,         // set of row indices
        int32_t *J,         // set of column indices
        int32_t *x,         // Set of values in int
        int32_t n,          // dimension of the matrix
        int32_t nz          // number of nonzeros in A (size of x, I,
                            // and J vectors)
    );
\end{verbatim}
} \end{mdframed}

\verb|SLIP_build_sparse_trip_int| builds a sparse matrix using triplet form
inputs, where the entry values are \verb|int32_t| type.

%-------------------------------------------------------------------------------
\cprotect\subsubsection{\verb|SLIP_build_sparse_trip_mpq|: build sparse matrix using triplet with \verb|mpq_t| entries}
\label{s:user:build_sparse_trip_mpq}
%-------------------------------------------------------------------------------

\begin{mdframed}[userdefinedwidth=6in]
{\footnotesize
\begin{verbatim}
    SLIP_info SLIP_build_sparse_trip_mpq
    (
        SLIP_sparse **A_handle,     // output matrix
        int32_t *I,         // set of row indices
        int32_t *J,         // set of column indices
        mpq_t *x,           // Set of values as rational numbers
        int32_t n,          // dimension of the matrix
        int32_t nz          // number of nonzeros in A (size of x, I
                            // and J vectors)
    );
\end{verbatim}
} \end{mdframed}

\verb|SLIP_build_sparse_trip_mpq| builds a sparse matrix using triplet form
inputs, where the entry values are \verb|mpq_t| type.

%-------------------------------------------------------------------------------
\cprotect\subsubsection{\verb|SLIP_build_sparse_trip_mpfr|: build sparse matrix using triplet with \verb|mpfr_t| entries}
\label{s:user:build_sparse_trip_mpfr}
%-------------------------------------------------------------------------------

\begin{mdframed}[userdefinedwidth=6in]
{\footnotesize
\begin{verbatim}
    SLIP_info SLIP_build_sparse_trip_mpfr
    (
        SLIP_sparse **A_handle,     // output matrix
        int32_t *I,         // set of row indices
        int32_t *J,         // set of column indices
        mpfr_t *x,          // Set of values as mpfr_t
        int32_t n,          // dimension of the matrix
        int32_t nz,         // number of nonzeros in A (size of x, I,
                            // and J vectors)
        SLIP_options *option// command options containing the prec for mpfr
    );
\end{verbatim}
} \end{mdframed}

\verb|SLIP_build_sparse_trip_mpfr| builds a sparse matrix using triplet form
inputs, where the entry values are \verb|mpfr_t| type.

%-------------------------------------------------------------------------------
\cprotect\subsubsection{\verb|SLIP_build_sparse_trip_mpz|: build sparse matrix using triplet with \verb|mpz_t| entries}
\label{s:user:build_sparse_trip_mpz}
%-------------------------------------------------------------------------------

\begin{mdframed}[userdefinedwidth=6in]
{\footnotesize
\begin{verbatim}
    SLIP_info SLIP_build_sparse_trip_mpz
    (
        SLIP_sparse **A_handle,     // output matrix
        int32_t *I,         // set of row indices
        int32_t *J,         // set of column indices
        mpz_t *x,           // Set of values in full precision int
        int32_t n,          // dimension of the matrix
        int32_t nz          // number of nonzeros in A (size of x, I,
                            // and J vectors)
    );
\end{verbatim}
} \end{mdframed}

\verb|SLIP_build_sparse_trip_mpz| builds a sparse matrix using triplet form
inputs, where the entry values are \verb|mpz_t| type.

%-------------------------------------------------------------------------------
\cprotect\subsubsection{\verb|SLIP_build_dense_double|: build dense matrix using 2D \verb|double| array}
\label{s:user:build_dense_double}
%-------------------------------------------------------------------------------

% TODO: describe in more detail.  Rename A and b (this is not A and b for Ax=b)

\begin{mdframed}[userdefinedwidth=6in]
{\footnotesize
\begin{verbatim}
    SLIP_info SLIP_build_dense_double
    (
        SLIP_dense **A_handle,      // Dense matrix to construct
        double **b,                 // Matrix of values as doubles
        int32_t m,                  // number of rows
        int32_t n                   // number of columns
    );
\end{verbatim}
} \end{mdframed}

\verb|SLIP_build_dense_double| builds a dense matrix using 2D \verb|double|
array.

%-------------------------------------------------------------------------------
\cprotect\subsubsection{\verb|SLIP_build_dense_int|: build dense matrix using 2D \verb|int32_t| array}
\label{s:user:build_dense_int}
%-------------------------------------------------------------------------------

\begin{mdframed}[userdefinedwidth=6in]
{\footnotesize
\begin{verbatim}
    SLIP_info SLIP_build_dense_int
    (
        SLIP_dense **A_handle,      // Dense matrix to construct
        int32_t **b,                // Matrix of values as ints
        int32_t m,                  // number of rows
        int32_t n                   // number of columns
    );
\end{verbatim}
} \end{mdframed}

\verb|SLIP_build_dense_int| builds a dense matrix using 2D \verb|int32_t|
array.

%-------------------------------------------------------------------------------
\cprotect\subsubsection{\verb|SLIP_build_dense_mpq|: build dense matrix using 2D \verb|mpq_t| array}
\label{s:user:build_dense_mpq}
%-------------------------------------------------------------------------------

\begin{mdframed}[userdefinedwidth=6in]
{\footnotesize
\begin{verbatim}
    SLIP_info SLIP_build_dense_mpq
    (
        SLIP_dense **A_handle,      // Dense matrix to construct
        mpq_t **b,                  // matrix of values as mpq_t
        int32_t m,                  // number of rows
        int32_t n                   // number of columns
    );
\end{verbatim}
} \end{mdframed}

\verb|SLIP_build_dense_mpq| builds a dense matrix using 2D \verb|mpq_t| array.

%-------------------------------------------------------------------------------
\cprotect\subsubsection{\verb|SLIP_build_dense_mpfr|: build dense matrix using 2D \verb|mpfr_t| array}
\label{s:user:build_dense_mpfr}
%-------------------------------------------------------------------------------

\begin{mdframed}[userdefinedwidth=6in]
{\footnotesize
\begin{verbatim}
    SLIP_info SLIP_build_dense_mpfr
    (
        SLIP_dense **A_handle,      // Dense matrix to construct
        mpfr_t **b,                 // Set of values as mpfr_t
        int32_t m,                  // number of rows
        int32_t n,                  // number of columns
        SLIP_options *option        // options with precision for mpfr
    );
\end{verbatim}
} \end{mdframed}

\verb|SLIP_build_dense_mpfr| builds a dense matrix using 2D \verb|mpfr_t|
array.

%-------------------------------------------------------------------------------
\cprotect\subsubsection{\verb|SLIP_build_dense_mpz|: build dense matrix using 2D \verb|mpz_t| array}
\label{s:user:build_dense_mpz}
%-------------------------------------------------------------------------------

\begin{mdframed}[userdefinedwidth=6in]
{\footnotesize
\begin{verbatim}
    SLIP_info SLIP_build_dense_mpz
    (
        SLIP_dense **A_handle,      // Dense matrix to construct
        mpz_t **b,                  // Set of values in full precision int.
        int32_t m,                  // number of rows
        int32_t n                   // number of columns
    );
\end{verbatim}
} \end{mdframed}

\verb|SLIP_build_dense_mpz| builds a dense matrix using 2D \verb|mpz_t| array.

%-------------------------------------------------------------------------------
\subsection{Primary Computational Routines}
%-------------------------------------------------------------------------------

These routines perform symbolic analysis prior to LU factorization, compute the
LU factorization of the matrix $A$, and solve $Ax=b$ using the LU factorization
of $A$.

%-------------------------------------------------------------------------------
\cprotect\subsubsection{\verb|SLIP_LU_analyze|: perform symbolic analysis}
\label{s:SLIP_LU_analyze}
%-------------------------------------------------------------------------------

\begin{mdframed}[userdefinedwidth=6in]
{\footnotesize
\begin{verbatim}
    SLIP_info SLIP_LU_analyze
    (
        SLIP_LU_analysis **S, // symbolic analysis
        SLIP_sparse *A,       // Input matrix
        SLIP_options *option  // Control parameters
    );
\end{verbatim}
} \end{mdframed}

\verb|SLIP_LU_analyze| performs the symbolic ordering for SLIP LU. Currently,
there are three options: user-defined order, COLAMD, or AMD, which are passed
in by \verb|SLIP_option *option|. For more details, users can refer to Section
\ref{ss:SLIP_options}.

The \verb|SLIP_LU_analysis| object is created by this function, and the value
of \verb|S| is ignored on input.  On output, \verb|S| is a pointer to the newly
created symbolic analysis object, or \verb|NULL| if a failure occurred.

The analysis \verb|S| is freed by \verb|SLIP_delete_LU_analysis|.

%-------------------------------------------------------------------------------
\cprotect\subsubsection{\verb|SLIP_LU_factorize|: perform LU factorization}
\label{ss:SLIP_LU_factorize}
%-------------------------------------------------------------------------------

\begin{mdframed}[userdefinedwidth=6in]
{\footnotesize
\begin{verbatim}
    SLIP_info SLIP_LU_factorize
    (
        // output:
        SLIP_sparse **L,         // lower triangular matrix
        SLIP_sparse **U,         // upper triangular matrix
        mpz_t **rhos,            // sequence of pivots
        int32_t **pinv,          // inverse row permutation
        // input:
        SLIP_sparse *A,         // matrix to be factored
        SLIP_LU_analysis *S,    // prior symbolic analysis
        SLIP_options *option    // command options
    );
\end{verbatim}
} \end{mdframed}

\verb|SLIP_LU_factorize| performs the SLIP LU factorization. This factorization
is done via $n$ (number of rows or columns of $A$) iterations of the sparse
REF triangular solve function. The overall factorization is $PAQ = LDU$.  This
routine allows the user to separate factorization and solve. For example codes,
please refer to either \verb|Demos/SLIPLU.c| or Section \ref{s:Using:expert}.

On input, \verb|L|, \verb|U|, \verb|rhos|, and \verb|pinv| are undefined.

On output, \verb|L| and \verb|U| are the lower and upper triangular matrices,
\verb|rhos| contains the sequence of pivots. The determinant of $A$ can be
obtained as \verb|rhos[n-1]|. \verb|pinv| contains the inverse row permutation
(that is, the row index in the permuted matrix $PA$. For the $i$th row in $A$,
\verb|pinv[i]| gives the row index in $PA$).

If an error occurs, \verb|L|, \verb|U|, \verb|rhos|, and \verb|pinv| are all
returned as \verb|NULL|.

%-------------------------------------------------------------------------------
\cprotect\subsubsection{\verb|SLIP_LU_solve|: solve the scaled linear system $LDUx=b$}
\label{ss:SLIP_LU_solve}
%-------------------------------------------------------------------------------

\begin{mdframed}[userdefinedwidth=6in]
{\footnotesize
\begin{verbatim}
    SLIP_info SLIP_LU_solve     //solves the linear system LDU x = b
    (
        // output:
        mpq_t **x,              // rational solution to the system
        // input:
        SLIP_dense *b,          // right hand side vector
        SLIP_sparse *L,         // lower triangular matrix
        SLIP_sparse *U,         // upper triangular matrix
        mpz_t *rhos,            // sequence of pivots
        int32_t *pinv           // row permutation
    );
\end{verbatim}
} \end{mdframed}

\verb|SLIP_LU_solve| obtains the solution to the scaled linear system $LDUx=b$
upon a successful factorization.  This function may be called after a
successful return from \verb|SLIP_LU_factorize|, which computes \verb|L|
\verb|U| \verb|rhos|, and \verb|pinv|. 

On input, \verb|mpq_t **x| should be allocated as a 2D array of same size as
\verb|b| using \verb|SLIP_create_mpq_mat| (see Section
\ref{ss:create_mpq_mat}).

Upon completion, \verb|x| contains the solution to the \textit{scaled} linear
system. Like some of some other routines discussed in this section, this
function is primarily for advanced users who might want intermediate
calculation results; thus for usage information please refer to either
\verb|Demos/SLIPLU.c| or Section \ref{s:Using:expert}.

%-------------------------------------------------------------------------------
\cprotect\subsubsection{\verb|SLIP_permute_x|: permute solution back to original form}
\label{ss:SLIP_permute_x}
%-------------------------------------------------------------------------------

\begin{mdframed}[userdefinedwidth=6in]
{\footnotesize
\begin{verbatim}
    SLIP_info SLIP_permute_x
    (
        mpq_t **x,            // Solution vector
        int32_t n,            // Size of solution vector
        int32_t numRHS,       // number of RHS vectors
        SLIP_LU_analysis *S   // symbolic analysis with the column ordering Q
    );
\end{verbatim}
} \end{mdframed}


\verb|SLIP_permute_x| permutes the solution vector(s) \verb|x| so that they are
with respect to the chosen column permutation (that is, this function computes
$Q \mathbf{x}$). The function is called upon successful return from
\verb|SLIP_LU_solve|.

%-------------------------------------------------------------------------------
\cprotect\subsubsection{\verb|SLIP_check_solution|: check if $A_{scaled}x=b_{scaled}$}
%-------------------------------------------------------------------------------

\begin{mdframed}[userdefinedwidth=6in]
{\footnotesize
\begin{verbatim}
    SLIP_info SLIP_check_solution
    (
        SLIP_sparse *A,           // input matrix
        mpq_t **x,                // solution vector
        SLIP_dense *b             // right hand side
    );
\end{verbatim}
} \end{mdframed}

\verb|SLIP_check_solution| checks the solution of the linear system. This
function returns either \verb|SLIP_CORRECT| or \verb|SLIP_INCORRECT|.

This function is provided simply for integrity or as troubleshoot code. It is
mostly not needed since the algorithm is designed to be exact. To use it
correctly, \verb|SLIP_check_solution| must be called before
\verb|SLIP_scale_x|. WARNING: \verb|SLIP_check_solution| could return
\verb|SLIP_INCORRECT| if it is called after \verb|SLIP_solve_double| (in
Section \ref{ss:SLIP_solve_double}), \verb|SLIP_solve_mpq| (in Section
\ref{ss:SLIP_solve_mpq}) or \verb|SLIP_solve_mpfr| (in Section
\ref{ss:SLIP_solve_mpfr}).


%-------------------------------------------------------------------------------
\cprotect\subsubsection{\verb|SLIP_scale_x|: scale solution with scaling factors of $A$ and $b$}
\label{ss:SLIP_scale_x}
%-------------------------------------------------------------------------------

\begin{mdframed}[userdefinedwidth=6in]
{\footnotesize
\begin{verbatim}
    SLIP_info SLIP_scale_x
    (
        mpq_t **x,              // Solution matrix
        SLIP_sparse *A,         // matrix A
        SLIP_dense *b           // right hand side
    );
\end{verbatim}
} \end{mdframed}

\verb|SLIP_scale_x| scales solution vector with scaling factors of $A$ and
$\mathbf{b}$. SLIP LU will scale the user's input matrix to ensure everything
is integer; thus, once the rational solution vector \verb|x| is obtained, it
must be properly scaled so that it is accurate. Again, this is mainly for
advanced users with needs for intermediate calculation results, thus for usage,
please refer to either \verb|Demos/SLIPLU.c| or Section \ref{s:Using:expert}.

%-------------------------------------------------------------------------------
\cprotect\subsubsection{\verb|SLIP_get_double_soln|: obtain solution in \verb|double| type}
\label{ss:get_double_soln}
%-------------------------------------------------------------------------------

\begin{mdframed}[userdefinedwidth=6in]
{\footnotesize
\begin{verbatim}
    SLIP_info SLIP_get_double_soln
    (
        double **x_doub,      // double soln of size n*numRHS to Ax = b
        mpq_t  **x_mpq,       // mpq solution to Ax = b. x is of size n*numRHS
        int32_t n,            // Dimension of A, number of rows of x
        int32_t numRHS        // Number of right hand side vectors
    ) ;
\end{verbatim}
} \end{mdframed}

\verb|SLIP_get_double_soln| converts the \verb|mpq_t**| solution vector
obtained from \verb|SLIP_LU_solve| and \verb|SLIP_permute_x| to
\verb|double**|. This process introduces round-off error.

On input, \verb|double **x_doub| should be allocated using
\verb|SLIP_create_double_mat| in Section \ref{ss:create_double_mat}.

%-------------------------------------------------------------------------------
\cprotect\subsubsection{\verb|SLIP_get_mpfr_soln|: obtain solution in \verb|mpfr_t| type}
\label{ss:get_mpfr_soln}
%-------------------------------------------------------------------------------

\begin{mdframed}[userdefinedwidth=6in]
{\footnotesize
\begin{verbatim}
    SLIP_info SLIP_get_mpfr_soln
    (
        mpfr_t **x_mpfr,      // mpfr solution of size n*numRHS to Ax = b
        mpq_t  **x_mpq,       // mpq solution of size n*numRHS to Ax = b.
        int32_t n,            // Dimension of A, number of rows of x
        int32_t numRHS        // Number of right hand side vectors
    );
\end{verbatim}
} \end{mdframed}

\verb|SLIP_get_mpfr_soln| converts the \verb|mpq_t**| solution vector obtained
from \verb|SLIP_LU_solve| and \verb|SLIP_permute_x| to \verb|mpfr_t**|. This
process introduces round-off error.

On input, \verb|mpfr_t **x_mpfr| should be allocated using
\verb|SLIP_create_mpfr_mat| in Section \ref{ss:create_mpfr_mat}.

%-------------------------------------------------------------------------------
\cprotect\subsubsection{\verb|SLIP_solve_double|: solve $Ax=b$ and return $x$ in \verb|double| type}
\label{ss:SLIP_solve_double}
%-------------------------------------------------------------------------------

\begin{mdframed}[userdefinedwidth=6in]
{\footnotesize
\begin{verbatim}
    SLIP_info SLIP_solve_double
    (
        double **x_doub,        // Solution vector stored as an double
        SLIP_sparse *A,         // CSC full precision matrix A
        SLIP_LU_analysis *S,    // Column ordering
        SLIP_dense *b,          // Right hand side vectrors
        SLIP_options *option    // Control parameters
    );
\end{verbatim}
} \end{mdframed}

\verb|SLIP_solve_double| solves the linear system $A\mathbf{x}=\mathbf{b}$ and
returns the solution as a matrix accurate to \verb|double| precision. This
function performs factorization, solving, permutation and scaling.  It must be
preceded by a call to the \verb|SLIP_LU_analysis| function, which constructs the
symbolic analysis object \verb|S|.

On output, \verb|x_doub| contains the solution to the linear system in
double precision and the function returns \verb|SLIP_OK|.

For a complete example, users can refer to \verb|Demos/example3.c|. Here is an
brief example of how to use this code:

{\small
\begin{verbatim}
    /* Create and populate A, b, and option */
    /* A has size of n-by-n, b has size of n-by-numRHS */
    SLIP_LU_analysis *S ;
    SLIP_LU_analyze(&S, A, option);
    double** x = SLIP_create_double_mat(n, numRHS);
    SLIP_solve_double(x, A, S, b, option);
\end{verbatim}
}

%-------------------------------------------------------------------------------
\cprotect\subsubsection{\verb|SLIP_solve_mpq|: solve $Ax=b$ and return $x$ in \verb|mpq_t| type}
\label{ss:SLIP_solve_mpq}
%-------------------------------------------------------------------------------

\begin{mdframed}[userdefinedwidth=6in]
{\footnotesize
\begin{verbatim}
    SLIP_info SLIP_solve_mpq
    (
        mpq_t **x_mpq,          // Solution vector stored as an mpq_t array
        SLIP_sparse *A,         // CSC form full precision matrix A
        SLIP_LU_analysis *S,    // Column ordering
        SLIP_dense *b,          // Right hand side vectrors
        SLIP_options *option    // Control parameters
    );
\end{verbatim}
} \end{mdframed}

\verb|SLIP_solve_mpq| solves the linear system $A\mathbf{x}=\mathbf{b}$ and
returns the solution as a matrix of \verb|mpq_t| numbers. This function
performs factorization, solving, permutation and scaling.  It must be preceded
by a call to the \verb|SLIP_LU_analysis| function, which constructs the
symbolic analysis object \verb|S|.

On output, \verb|x_mpq| contains the exact solution to the linear system
as \verb|mpq_t| numbers and the function returns \verb|SLIP_OK|.

For a complete example, users can refer to \verb|Demos/example2.c|. Here is an
brief example of how to use this code:

{\small
\begin{verbatim}
    /* Create and populate A, b, and option */
    /* A has size of n-by-n, b has size of n-by-numRHS */
    SLIP_LU_analysis *S ;
    SLIP_LU_analyze(&S, A, option);
    mpq_t** x = SLIP_initialize_mpq_mat(n, numRHS);
    SLIP_solve_mpq(x, A, S, b, option);
\end{verbatim}
}

%-------------------------------------------------------------------------------
\cprotect\subsubsection{\verb|SLIP_solve_mpfr|: solve $Ax=b$ and return $x$ in \verb|mpfr_t| type}
\label{ss:SLIP_solve_mpfr}
%-------------------------------------------------------------------------------

\begin{mdframed}[userdefinedwidth=6in]
{\footnotesize
\begin{verbatim}
    SLIP_info SLIP_solve_mpfr
    (
        mpfr_t **x_mpfr,        // Solution vector stored as an mpfr_t array
        SLIP_sparse *A,         // CSC form full precision matrix A
        SLIP_LU_analysis *S,    // Column ordering
        SLIP_dense *b,          // Right hand side vectrors
        SLIP_options *option    // Control parameters
    );
\end{verbatim}
} \end{mdframed}


\verb|SLIP_solve_mpq| solves the linear system $A\mathbf{x}=\mathbf{b}$ and
returns the solution as a matrix of \verb|mpfr_t| numbers. This function
performs factorization, solving, permutation and scaling.  It must be preceded
by a call to the \verb|SLIP_LU_analysis| function, which constructs the
symbolic analysis object \verb|S|.

On output, \verb|x_mpfr| contains the exact solution to the linear system
as \verb|mpfr_t| numbers and the function returns \verb|SLIP_OK|.

Here is an brief example of how to use this code:

{\small
\begin{verbatim}
    /* Create and populate A, b, and option */
    /* A has size of n-by-n, b has size of n-by-numRHS */
    SLIP_LU_analysis *S ;
    SLIP_LU_analyze(&S, A, option);
    option->prec = 128; // Quad
    mpfr_t** x = SLIP_create_mpfr_mat(nrows, numRHS, option);
    SLIP_solve_mpfr(x, A, S, b, option);
\end{verbatim}
}

%-------------------------------------------------------------------------------
\subsection{Miscellaneous Routines (TODO rename this)}
\label{s:miscellaneous_routine}
%-------------------------------------------------------------------------------

This section contains miscellaneous routines that may be of interest to the
user.
% TODO:
{\bf TODO: this is a confusing name for this section.  And ``may be of
interest'' is misleading, since functions like \verb|SLIP_create_double_mat| is
used in many places.}
% JC: These are the functions that might need to move to internal?  As we
% discussed in the email, we can make the *create_*_mat/array functions
% internal to simplify the interface. In this way, users only need to pass
% pointers that are initially NULL (e.g., &x to SLIP_solve_*, or &rhos and &pinv
% to SLIP_LU_solve). However, since user would still need to destroy the mat(s)
% and array(s), should we keep the *delete_*_mat/array user-callable?

%-------------------------------------------------------------------------------
\cprotect\subsubsection{\verb|SLIP_create_double_mat|: create a $m$-by-$n$ \verb|double| matrix} \label{ss:create_double_mat}
%-------------------------------------------------------------------------------

\begin{mdframed}[userdefinedwidth=6in]
{\footnotesize
\begin{verbatim}
    double** SLIP_create_double_mat
    (
        int32_t m,     // number of rows (must be > 0)
        int32_t n      // number of columns (must be > 0)
    );
\end{verbatim}
} \end{mdframed}

\verb|SLIP_create_double_mat| allocates a \verb|double| matrix of size $m
\times n$ and sets each entry equal to zero, where $A[i][j]$ is the ($i,j$)th
entry. $A[i]$ is a pointer to row $i$, of size $n$. \verb|NULL| is returned if
$m \le 0 $ or $n\le 0$ or out of memory.

% TODO: I don't understand this data structure. A is just double ** pointer?
% So where does it store m and n?

% JC: this is only used to create the solution in double format, therefore,
% the size would be same as RHS

% TD: That is not what I'm asking.  A bare double ** pointer has no way to
% store the size of the object.  I'm asking if we should encapsulate the
% double ** pointer inside a struct for dense matrix, just like we do
% for SLIP_sparse.

%-------------------------------------------------------------------------------
\cprotect\subsubsection{\verb|SLIP_delete_double_mat|: delete a $m$-by-$n$ \verb|double| matrix}
%-------------------------------------------------------------------------------

\begin{mdframed}[userdefinedwidth=6in]
{\footnotesize
\begin{verbatim}
    void SLIP_delete_double_mat
    (
        double*** A,   // dense matrix
        int32_t m,     // number of rows of A
        int32_t n      // number of columns of A
    );
\end{verbatim}
} \end{mdframed}

\verb|SLIP_delete_double_mat| frees the memory associated with a \verb|double|
matrix of size $m \times n$, and sets \verb|**A=NULL|.

% TODO:  fix dense matrix data structure

\begin{verbatim}
TODO: any time a code needs a triple star pointer, something is wrong.
This data structure should be redesigned. Also for integer matrix.
        double*** A,   // dense matrix
\end{verbatim}

%-------------------------------------------------------------------------------
\cprotect\subsubsection{\verb|SLIP_create_int_mat|: create a $m$-by-$n$ \verb|int32_t| matrix}
%-------------------------------------------------------------------------------

\begin{mdframed}[userdefinedwidth=6in]
{\footnotesize
\begin{verbatim}
    int32_t** SLIP_create_int_mat
    (
        int32_t m,     // number of rows (must be > 0)
        int32_t n      // number of columns (must be > 0)
    );
\end{verbatim}
} \end{mdframed}

\verb|SLIP_create_int_mat| allocates a \verb|int32_t| matrix of size $m \times
n$ and sets each entry equal to zero, where $A[i][j]$ is the ($i,j$)th entry.
$A[i]$ is a pointer to row $i$, of size $n$. \verb|NULL| is returned if
$m \le 0 $ or $n\le 0$ or out of memory.


%-------------------------------------------------------------------------------
\cprotect\subsubsection{\verb|SLIP_delete_int_mat|: delete a $m$-by-$n$ \verb|int32_t| matrix}
%-------------------------------------------------------------------------------

\begin{mdframed}[userdefinedwidth=6in]
{\footnotesize
\begin{verbatim}
    void SLIP_delete_int_mat
    (
        int32_t*** A,  // dense matrix
        int32_t m,     // number of rows
        int32_t n      // number of columns
    );
\end{verbatim}
} \end{mdframed}

\verb|SLIP_delete_int_mat| frees the memory associated with a \verb|int32_t|
matrix of size $m \times n$, and sets \verb|**A=NULL|.

%-------------------------------------------------------------------------------
\cprotect\subsubsection{\verb|SLIP_create_mpfr_mat|: create a $m$-by-$n$ \verb|mpfr_t| matrix}
\label{ss:create_mpfr_mat}
%-------------------------------------------------------------------------------

\begin{mdframed}[userdefinedwidth=6in]
{\footnotesize
\begin{verbatim}
    mpfr_t** SLIP_create_mpfr_mat
    (
        int32_t m,     // number of rows (must be > 0)
        int32_t n,     // number of columns (must be > 0)
        SLIP_options *option  // command options containing the prec for mpfr
    );
\end{verbatim}
} \end{mdframed}


\verb|SLIP_create_mpfr_mat| allocates a \verb|mpfr_t| matrix of size $m \times
n$ and sets each entry equal to zero, where $A[i][j]$ is the ($i,j$)th entry.
$A[i]$ is a pointer to row $i$, of size $n$. The floating point precision
associated with each entry is given by \verb|option->prec|. \verb|NULL| is returned if
$m \le 0 $ or $n\le 0$ or out of memory.


%-------------------------------------------------------------------------------
\cprotect\subsubsection{\verb|SLIP_delete_mpfr_mat|: delete a $m$-by-$n$ \verb|mpfr_t| matrix}
%-------------------------------------------------------------------------------

\begin{mdframed}[userdefinedwidth=6in]
{\footnotesize
\begin{verbatim}
    void SLIP_delete_mpfr_mat
    (
        mpfr_t ***A,   // Dense mpfr matrix
        int32_t m,     // number of rows of A
        int32_t n      // number of columns of A
    );
\end{verbatim}
} \end{mdframed}

\begin{verbatim}
TODO: any time a code needs a triple star pointer, something is wrong.
This data structure is terribly confusion:
        mpfr_t ***A,   // Dense mpfr matrix
\end{verbatim}

\verb|SLIP_delete_mpfr_mat| frees the memory associated with a \verb|mpfr_t|
matrix of size $m \times n$, and sets \verb|**A=NULL|.

%-------------------------------------------------------------------------------
\cprotect\subsubsection{\verb|SLIP_create_mpq_mat|: create a $m$-by-$n$ \verb|mpq_t| matrix}
\label{ss:create_mpq_mat}
%-------------------------------------------------------------------------------

\begin{mdframed}[userdefinedwidth=6in]
{\footnotesize
\begin{verbatim}
    mpq_t** SLIP_create_mpq_mat
    (
        int32_t m,     // number of rows (must be > 0)
        int32_t n      // number of columns (must be > 0)
    );
\end{verbatim}
} \end{mdframed}

\verb|SLIP_create_mpq_mat| allocates a \verb|mpq_t| matrix of size $m \times n$
and sets each entry equal to zero, where $A[i][j]$ is the ($i,j$)th entry.
$A[i]$ is a pointer to row $i$, of size $n$. \verb|NULL| is returned if
$m \le 0 $ or $n\le 0$ or out of memory.


%-------------------------------------------------------------------------------
\cprotect\subsubsection{\verb|SLIP_delete_mpq_mat|: delete a $m$-by-$n$ \verb|mpq_t| matrix}
%-------------------------------------------------------------------------------

\begin{mdframed}[userdefinedwidth=6in]
{\footnotesize
\begin{verbatim}
    void SLIP_delete_mpq_mat
    (
        mpq_t***A,     // dense mpq matrix
        int32_t m,     // number of rows of A
        int32_t n      // number of columns of A
    );
\end{verbatim}
} \end{mdframed}


\verb|SLIP_delete_mpq_mat| frees the memory associated with a \verb|mpq_t|
matrix of size $m \times n$, and sets \verb|**A=NULL|.

%-------------------------------------------------------------------------------
\cprotect\subsubsection{\verb|SLIP_create_mpz_mat|: create a $m$-by-$n$ \verb|mpz_t| matrix}
%-------------------------------------------------------------------------------

\begin{mdframed}[userdefinedwidth=6in]
{\footnotesize
\begin{verbatim}
    mpz_t** SLIP_create_mpz_mat
    (
        int32_t m,     // number of rows (must be > 0)
        int32_t n      // number of columns (must be > 0)
    );
\end{verbatim}
} \end{mdframed}

\verb|SLIP_create_mpz_mat| allocates a \verb|mpz_t| matrix of size $m \times n$
and sets each entry equal to zero, where $A[i][j]$ is the ($i,j$)th entry.
$A[i]$ is a pointer to row $i$, of size $n$. \verb|NULL| is returned if
$m \le 0 $ or $n\le 0$ or out of memory.


%-------------------------------------------------------------------------------
\cprotect\subsubsection{\verb|SLIP_delete_mpz_mat|: delete a $m$-by-$n$ \verb|mpz_t| matrix}
%-------------------------------------------------------------------------------

\begin{mdframed}[userdefinedwidth=6in]
{\footnotesize
\begin{verbatim}
    void SLIP_delete_mpz_mat
    (
        mpz_t ***A,     // The dense mpz matrix
        int32_t m,      // number of rows of A
        int32_t n       // number of columns of A
    );
\end{verbatim}
} \end{mdframed}

\verb|SLIP_delete_mpz_mat| frees the memory associated with a \verb|mpz_t|
matrix of size $m \times n$, and sets \verb|**A=NULL|.

%-------------------------------------------------------------------------------
\cprotect\subsubsection{\verb|SLIP_create_mpfr_array|: create a \verb|mpfr_t| of length $n$}
%-------------------------------------------------------------------------------

\begin{mdframed}[userdefinedwidth=6in]
{\footnotesize
\begin{verbatim}
    mpfr_t* SLIP_create_mpfr_array
    (
        int32_t n,     // size of the array (must be > 0)
        SLIP_options *option  // command options containing the prec for mpfr
    );
\end{verbatim}
} \end{mdframed}

\verb|SLIP_create_mpfr_array| allocates a \verb|mpfr_t| matrix of length $n$
and sets each entry equal to zero, where  $A[i]$ is an entry of type
\verb|mpfr_t|. The floating point precision associated with each entry is given
by \verb|option->prec|. \verb|NULL| is returned if
$n\le 0$ or out of memory.


%-------------------------------------------------------------------------------
\cprotect\subsubsection{\verb|SLIP_delete_mpfr_array|: delete a \verb|mpfr_t| of length $n$}
%-------------------------------------------------------------------------------

\begin{mdframed}[userdefinedwidth=6in]
{\footnotesize
\begin{verbatim}
    void SLIP_delete_mpfr_array
    (
        mpfr_t** x,    // mpfr array to be deleted
        int32_t n      // size of x
    );
\end{verbatim}
} \end{mdframed}

\verb|SLIP_delete_mpfr_array| frees the memory associated with a \verb|mpfr_t|
array of size $n$, and sets \verb|*x=NULL|.

%-------------------------------------------------------------------------------
\cprotect\subsubsection{\verb|SLIP_create_mpq_array|: create a \verb|mpq_t| of length $n$}
%-------------------------------------------------------------------------------

\begin{mdframed}[userdefinedwidth=6in]
{\footnotesize
\begin{verbatim}
    mpq_t* SLIP_create_mpq_array
    (
        int32_t n      // size of the array (must be > 0)
    );
\end{verbatim}
} \end{mdframed}


\verb|SLIP_create_mpq_array| allocates a \verb|mpq_t| matrix of length $n$ and
sets each entry equal to zero, where  $A[i]$ is an entry of type \verb|mpq_t|.
 \verb|NULL| is returned if
$n\le 0$ or out of memory.

%-------------------------------------------------------------------------------
\cprotect\subsubsection{\verb|SLIP_delete_mpq_array|: delete a \verb|mpq_t| of length $n$}
%-------------------------------------------------------------------------------

\begin{mdframed}[userdefinedwidth=6in]
{\footnotesize
\begin{verbatim}
    void SLIP_delete_mpq_array
    (
        mpq_t** x,     // mpq array to be deleted
        int32_t n      // size of x
    );
\end{verbatim}
} \end{mdframed}


\verb|SLIP_delete_mpq_array| frees the memory associated with a \verb|mpq_t|
array of size $n$, and sets \verb|*x=NULL|.

%-------------------------------------------------------------------------------
\cprotect\subsubsection{\verb|SLIP_create_mpz_array|: create a \verb|mpz_t| of length $n$}
\label{ss:create_mpz_array}
%-------------------------------------------------------------------------------

\begin{mdframed}[userdefinedwidth=6in]
{\footnotesize
\begin{verbatim}
    mpz_t* SLIP_create_mpz_array
    (
        int32_t n      // Size of x (must be > 0)
    );
\end{verbatim}
} \end{mdframed}

\verb|SLIP_create_mpz_array| allocates a \verb|mpz_t| matrix of length $n$ and
sets each entry equal to zero, where  $A[i]$ is an entry of type \verb|mpz_t|.
 \verb|NULL| is returned if
$n\le 0$ or out of memory.

%-------------------------------------------------------------------------------
\cprotect\subsubsection{\verb|SLIP_delete_mpz_array|: delete a \verb|mpz_t| of length $n$}
%-------------------------------------------------------------------------------

\begin{mdframed}[userdefinedwidth=6in]
{\footnotesize
\begin{verbatim}
    void SLIP_delete_mpz_array
    (
        mpz_t ** x,     // mpz array to be deleted
        int32_t n       // Size of x
    );
\end{verbatim}
} \end{mdframed}

\verb|SLIP_delete_mpz_array| frees the memory associated with a \verb|mpz_t|
array of size $n$, and sets \verb|*x=NULL|.

%-------------------------------------------------------------------------------
\cprotect\subsubsection{\verb|SLIP_spok|: check and print a \verb|SLIP_sparse| matrix}
%-------------------------------------------------------------------------------

\begin{mdframed}[userdefinedwidth=6in]
{\footnotesize
\begin{verbatim}
    SLIP_info SLIP_spok  // returns a SLIP_LU status code
    (
        SLIP_sparse *A,     // matrix to check
        int32_t print_level // 0: print nothing, 1: just errors,
                            // 2: terse, 3: all
    ) ;
\end{verbatim}
} \end{mdframed}

\verb|SLIP_spok| check the validity of a \verb|SLIP_sparse| matrix in compressed-sparse column form.  Derived from \verb|SuiteSparse/MATLAB_TOOLS/spok|.

%-------------------------------------------------------------------------------
\subsection{SLIP LU GMP and MPFR Wrappers}
%-------------------------------------------------------------------------------

SLIP LU provides a wrapper class for all GMP and MPFR functions used by SLIP
LU.  The wrapper class provides error-handling for out-of-memory conditions
that are not handled by the GMP and MPFR libraries.  These wrapper functions
are used inside all SLIP LU functions, wherever any GMP or MPFR functions are
used.  These functions may also be called by the end-user application.

Each wrapped function has the same name as its corresponding GMP/MPFR function
with the added prefix \verb|SLIP_|. For example, the default GMP function
\verb|mpz_mul| is changed to \verb|SLIP_mpz_mul|. Each SLIP GMP/MPFR function
returns \verb|SLIP_OK| if successful or the correct error code if not. The
following table gives a brief list of each currently covered SLIP GMP/MPFR
function. For a detailed description of each function, please refer to
\verb|SLIP_LU/Source/SLIP_gmp.c|.

If additional GMP and MPFR functions are needed in the end-user application,
this wrapper mechanism can be extended to those functions.  Below, we give
instructions on how to do this.

Given a GMP function \verb|void gmpfunc(TYPEa a, TYPEb b, ...)|, where
\verb|TYPEa| and \verb|TYPEb| can be GMP type data (\verb|mpz_t|,
\verb|mpq_t| and \verb|mpfr_t|, for example) or non-GMP type data (\verb|int|,
\verb|double|, for example), and they need not to be the same. In order to
apply our wrapper to a new function, one can create it as follows:

\begin{mdframed}[userdefinedwidth=6in]
{\footnotesize
\begin{verbatim}

SLIP_info SLIP_gmpfunc
(
    TYPEa a,
    TYPEb b,
    ...
)
{
    // Start the GMP Wrappter
    // uncomment one of the followings that meets the needs
    // If this function is not modifying any GMP type variable, then use
    //SLIP_GMP_WRAPPER_START;
    // If this function is modifying mpz_t type (say TYPEa = mpz_t), then use
    //SLIP_GMPZ_WRAPPER_START(a);
    // If this function is modifying mpq_t type (say TYPEa = mpq_t), then use
    //SLIP_GMPQ_WRAPPER_START(a);
    // If this function is modifying mpz_t type (say TYPEa = mpz_t), then use
    //SLIP_GMPFR_WRAPPER_START(a);

    // Call the GMP function
    gmpfunc(a,b,...);

    //Finish the wrapper and return ok if successful.
    SLIP_GMP_WRAPPER_FINISH;
    return SLIP_OK;
}
\end{verbatim}
} \end{mdframed}

Note that, other than \verb|SLIP_mpfr_fprintf|, \verb|SLIP_gmp_fprintf|,
\verb|SLIP_gmp_printf| and \verb|SLIP_gmp_fscanf|, all of the wrapped GMP/MPFR
functions always return \verb|SLIP_info| to the caller. Therefore, for some
GMP/MPFR functions that have their own return value.  For example, fo
\verb|int mpq_cmp(const mpq_t a, const mpq_t b)|, the return value becomes a
parameter of the wrapped function. In general, a GMP/MPFR function in the form
of \verb|TYPEr gmpfunc(TYPEa a, TYPEb b, ...)|, users can create the wrapped
function as follows:

\begin{mdframed}[userdefinedwidth=6in]
{\footnotesize
\begin{verbatim}
SLIP_info SLIP_gmpfunc
(
    TYPEr *r,        // return value of the GMP/MPFR function
    TYPEa a,
    TYPEb b,
    ...
)
{
    // Start the GMP Wrappter
    // uncomment one of the followings that meets the needs
    //SLIP_GMP_WRAPPER_START;
    //SLIP_GMPZ_WRAPPER_START(a);
    //SLIP_GMPQ_WRAPPER_START(a);
    //SLIP_GMPFR_WRAPPER_START(a);

    // Call the GMP function
    *r = gmpfunc(a,b,...);

    //Finish the wrapper and return ok if successful.
    SLIP_GMP_WRAPPER_FINISH;
    return SLIP_OK;
}
\end{verbatim}
} \end{mdframed}

\newpage
\thispagestyle{empty}
{\scriptsize
% \begin{table*}[htbp]
\begin{center}
\begin{tabular}{|l|l|l|}
\hline
MPFR Function & \verb|SLIP_MPFR| Function & Description \\
\hline\hline
\verb|n = mpfr_fprintf(fp, format, ...)|
    & \verb|n = SLIP_mpfr_fprintf(fp, format, ...)|
    & Print format to file fp \\ \hline
\verb|mpfr_init2(x, size)|
    & \verb|SLIP_mpfr_init2(x, size)|
    & Initialize x with size bits \\ \hline
\verb|mpfr_set(x, y, rnd)|
    & \verb|SLIP_mpfr_set(x, y, rnd)|
    & $x = y$ \\ \hline
\verb|mpfr_set_d(x, y, rnd)|
    & \verb|SLIP_mpfr_set_d(x, y, rnd)|
    & $x = y$ (double) \\ \hline
\verb|mpfr_set_q(x, y, rnd)|
    & \verb|SLIP_mpfr_set_q(x, y, rnd)|
    & $x = y$ (mpq) \\ \hline
\verb|mpfr_set_z(x, y, rnd)|
    & \verb|SLIP_mpfr_set_z(x, y, rnd)|
    & $x = y$ (mpz) \\ \hline
\verb|mpfr_get_z(x, y, rnd)|
    & \verb|SLIP_mpfr_get_z(x, y, rnd)|
    & (mpz) $x = y$\\ \hline
\verb|x = mpfr_get_d(y, rnd)|
    & \verb|SLIP_mpfr_get_d(x, y, rnd)|
    & (double) $x = y$\\ \hline
\verb|mpfr_mul(x, y, z, rnd)|
    & \verb|SLIP_mpfr_mul(x, y, z, rnd)|
    & $x = y*z$ \\ \hline
\verb|mpfr_mul_d(x, y, z, rnd)|
    & \verb|SLIP_mpfr_mul_d(x, y, z, rnd)|
    & $x = y*z$ \\ \hline
\verb|mpfr_div_d(x, y, z, rnd)|
    & \verb|SLIP_mpfr_div_d(x, y, z, rnd)|
    & $x = y/z$ \\ \hline
\verb|mpfr_ui_pow_ui(x, y, z, rnd)|
    & \verb|SLIP_mpfr_ui_pow_ui(x, y, z, rnd)|
    & $x = y^z$ \\ \hline
\verb|mpfr_log2(x, y, rnd)|
    & \verb|SLIP_mpfr_log2(x, y, rnd )|
    & $x = \log_2 (y)$ \\ \hline
\verb|mpfr_free_cache()|
    & \verb|SLIP_mpfr_free_cache()|
    & Free cache after log2 \\ \hline
% \end{tabular}
% \end{center}
% \end{table*}
% 
% 
% \begin{table*}[htbp]
% \begin{center}
% \begin{tabular}{|l|l|l|}
\hline
GMP Function & \verb|SLIP_GMP| Function & Description \\
\hline\hline
\verb|n = gmp_fprintf(fp, format, ...)|
    & \verb|n = SLIP_gmp_fprintf(fp, format, ...)| 
    & Print format to file fp\\ \hline
\verb|n = gmp_printf(format, .. )|
    & \verb|n = SLIP_gmp_printf(format, ...)|
    & Print to screen \\ \hline
\verb|n = gmp_fscanf(fp, format, ...)|
    & \verb|n = SLIP_gmp_fscanf(fp, format, ...)|
    & Read from file fp \\ \hline
\verb|mpz_init(x)|
    & \verb|SLIP_mpz_init(x)|
    & Initialize x \\ \hline
\verb|mpz_init2(x, size)|
    & \verb|SLIP_mpz_init2(x, size)|
    & Initialize x to size bits \\ \hline
\verb|mpz_set(x, y)|
    & \verb|SLIP_mpz_set(x, y)| 
    & $x = y$ (mpz) \\ \hline
\verb|mpz_set_ui(x, y)|
    & \verb|SLIP_mpz_set_ui(x, y)|
    & $x = y$ (signed int) \\ \hline
\verb|mpz_set_si(x, y)|
    & \verb|SLIP_mpz_set_si(x, y)|
    & $x = y$ (unsigned int) \\ \hline
\verb|mpz_set_d(x, y)|
    & \verb|SLIP_mpz_set_d(x, y)|
    & $x = y$ (double)\\ \hline
\verb|x = mpz_get_d(y)|
    & \verb|SLIP_mpz_get_d(x, y)|
    & $x = y$ (double out) \\ \hline
\verb|mpz_set_q(x, y)|
    & \verb|SLIP_mpz_set_q(x, y)|
    & $x = y$ (mpq) \\ \hline
\verb|mpz_mul(x, y, z)|
    & \verb|SLIP_mpz_mul(x, y, z)|
    & $x = y*z$ \\ \hline
\verb|mpz_add(x, y, z)|
    & \verb|SLIP_mpz_add(x, y, z)|
    & $x = y+z$ \\ \hline
\verb|mpz_addmul(x, y, z)|
    & \verb|SLIP_mpz_addmul(x, y, z)|
    & $x = x+y*z$ \\ \hline
\verb|mpz_submul(x, y, z)|
    & \verb|SLIP_mpz_submul(x, y, z)|
    & $x = x-y*z$ \\ \hline
\verb|mpz_divexact(x, y, z)|
    & \verb|SLIP_mpz_divexact(x, y, z)|
    & $x = y/z$ \\ \hline
\verb|gcd = mpz_gcd(x, y)|
    & \verb|SLIP_mpz_gcd(gcd, x, y)|
    & $gcd = gcd(x,y)$\\ \hline
\verb|lcm = mpz_lcm(x, y)|
    & \verb|SLIP_mpz_lcm(lcm, x, y)|
    & $lcm = lcm(x,y)$ \\ \hline
\verb|mpz_abs(x, y)|
    & \verb|SLIP_mpz_abs(x, y)|
    & $x = |y|$ \\ \hline
\verb|r = mpz_cmp(x, y)|
    & \verb|SLIP_mpz_cmp(r, x, y)|
    & $r = 0$ if $x=y$\\&& $r\neq 0$  if $x\neq y$ \\ \hline
\verb|r = mpz_cmpabs(x, y)|
    & \verb|SLIP_mpz_cmpabs(r, x, y)|
    & $r = 0$ if $|x|=|y|$\\&&  $r\neq 0$  if $|x|\neq |y|$\\ \hline
\verb|r = mpz_cmp_ui(x, y)|
    & \verb|SLIP_mpz_cmp_ui(r, x, y)|
    & $r = 0$ if $x=y$\\&&  $r\neq 0$  if $x\neq y$ \\ \hline
\verb|sgn = mpz_sgn(x)|
    & \verb|SLIP_mpz_sgn(sgn, x)|
    & $sgn = 0$ if $x = 0$ \\ \hline
\verb|size = mpz_sizeinbase(x, base)|
    & \verb|SLIP_mpz_sizeinbase(size, x, base)|
    & size of x in base \\ \hline
\verb|mpq_init(x)|
    & \verb|SLIP_mpq_init(x)|
    & Initialize x \\ \hline
\verb|mpq_set(x, y)|
    & \verb|SLIP_mpq_set(x, y)|
    & $x = y$ \\ \hline
\verb|mpq_set_z(x, y)|
    & \verb|SLIP_mpq_set_z(x, y)|
    & $x = y$ (mpz) \\ \hline
\verb|mpq_set_d(x, y)|
    & \verb|SLIP_mpq_set_d(x, y)|
    & $x=y$ (double) \\ \hline
\verb|mpq_set_ui(x, y, z)|
    & \verb|SLIP_mpq_set_ui(x, y, z)|
    & $x = y/z$ (unsigned int) \\ \hline
\verb|mpq_set_num(x, y)|
    & \verb|SLIP_mpq_set_num(x, y)|
    & $num(x) = y$ \\ \hline
\verb|mpq_set_den(x, y)|
    & \verb|SLIP_mpq_set_den(x, y)|
    & $den(x) = y$ \\ \hline
\verb|mpq_get_den(x, y)|
    & \verb|SLIP_mpq_get_den(x, y)|
    & $x = den(y)$ \\ \hline
\verb|x = mpq_get_d(y)|
    & \verb|SLIP_mpq_get_d(x, y)|
    & (double) $x = y$ \\ \hline
\verb|mpq_abs(x, y)|
    & \verb|SLIP_mpq_abs(x, y)|
    & $x = |y|$ \\ \hline
\verb|mpq_add(x, y, z)|
    & \verb|SLIP_mpq_add(x, y, z)|
    & $x = y+z$ \\ \hline
\verb|mpq_mul(x, y, z)|
    & \verb|SLIP_mpq_mul(x, y, z)|
    & $x = y*z$ \\ \hline
\verb|mpq_div(x, y, z)|
    & \verb|SLIP_mpq_div(x, y, z)|
    & $x = y/z$ \\ \hline
\verb|r = mpq_cmp(x, y)|
    & \verb|SLIP_mpq_cmp(r, x, y)|
    & $r = 0$ if $x=y$\\&&  $r\neq 0$ if $x\neq y$ \\ \hline
\verb|r = mpq_cmp_ui(x, n, d)|
    & \verb|SLIP_mpq_cmp_ui(r, x, n, d)|
    & $r = 0$ if $x=n/d$\\&& $r\neq 0$ if $x\neq n/d$ \\ \hline
\verb|r = mpq_equal(x, y)|
    & \verb|SLIP_mpq_equal(r, x, y)|
    & $r = 0$ if $x=y$\\&&  $r\neq 0$ if $x\neq y$ \\ \hline
\end{tabular}
\end{center}
% \end{table*}
}

%-------------------------------------------------------------------------------
\cprotect\section{Using SLIP LU in C} \label{s:Using}
%-------------------------------------------------------------------------------

Using SLIP LU in C has four steps:

\begin{enumerate}
\item initialize and populate data structures,
\item perform symbolic analysis,
\item factorize the matrix $A$ and solve the linear
system for each $\mathbf{b}$ vector, and
\item free all used memory and finalize.
\end{enumerate}

Steps 1 and 2 are discussed in Subsections \ref{s:Using:init} and
\ref{s:Using:symb}. Factorizing $A$ and solving the linear $A \mathbf{x} =
\mathbf{b}$ can be done in one of two ways. If the user is only interested in
obtaining the solution vector $\mathbf{x}$, SLIP LU provides a simple interface
for this purpose which is discussed in Section \ref{s:Using:simple}.
Alternatively, if the user wants the actual $L$ and $U$ factors, please refer
to Section \ref{s:Using:expert}. Finally, step 4 is discussed in Section
\ref{s:Using:free}. For the remainder of this section, \verb|n| will indicate
the dimension of $A$ (that is, $A \in \mathbb{Z}^{n \times n}$) and
\verb|numRHS| will indicate the number of right hand side vectors being solved
(that is, if \verb|numRHS|$= r$, then $\mathbf{b} \in \mathbb{Z}^{n \times
r})$.

%-------------------------------------------------------------------------------
\cprotect\subsection{SLIP LU Initialization and Population of Data Structures}
\label{s:Using:init}
%-------------------------------------------------------------------------------

This section discusses how to initialize and populate the global data
structures required for SLIP LU.

%-------------------------------------------------------------------------------
\subsubsection{Initializing the Environment}
%-------------------------------------------------------------------------------

SLIP LU is built upon the GNU GMP library \cite{granlund2015gnu} and provides
wrappers to all GMP functions it uses.  This allows SLIP LU to properly handle
memory management failures, which GMP does not handle.  It may also allow the
user to not need any direct access to the GMP library.  To enable this
mechanism, SLIP LU requires initialization.  The following must be done before
using any other SLIP LU function:

\verb|SLIP_initialize();|

%-------------------------------------------------------------------------------
\subsubsection{Initializing Data Structures}
\label{ss:init}
%-------------------------------------------------------------------------------

SLIP LU assumes four specific input options for all functions. These are:

\begin{itemize}
\item \verb|SLIP_sparse* A|: \verb|A| contains the user's input matrix. If the
input matrix was already an integer matrix, \verb|A| is the user's input and
\verb|A->scale|=1. Otherwise, the input matrix is not integer and \verb|A|
contains the user's scaled input matrix.

\item \verb|SLIP_LU_analysis* S|: \verb|S| contains the column permutation used
for $A$ as well as guesses for the number of nonzeros in $L$ and $U$.

\item \verb|SLIP_options* option|: \verb|option| contains various control
options for the factorization including column ordering used, pivot selection
scheme, and others. For a full list of the contents of the \verb|SLIP_options|
structure, please refer to Section \ref{ss:SLIP_options}.

\item \verb|SLIP_dense* b|: \verb|b| contains the user's right hand side
vector(s). If the input right hand side vectors were already integer, \verb|b|
contains them directly and \verb|b->scale|=1. Otherwise, \verb|b| is the scaled
input right hand side vector(s).
\end{itemize}

%-------------------------------------------------------------------------------
\subsubsection{Populating Data Structures}
%-------------------------------------------------------------------------------

Of the four data structures discussed in Section~\ref{ss:init}, \verb|S| is
constructed during symbolic analysis (Section \ref{s:Using:symb}), \verb|option|
is initialized to default values and can be modified if the user desires
(please refer to Section \ref{ss:SLIP_options} for the contents of
\verb|option|) and \verb|A| and \verb|b| are
% TODO: this is not correct:
populated by the user.
TODO: no, we have functions that create A and b.

SLIP LU allows the input numerical data for \verb|A| and \verb|b| to come in
one of 5 options: \verb|int32_t|, \verb|double|, \verb|mpfr_t|, \verb|mpq_t|,
and \verb|mpz_t|. Moreover, \verb|A| can be stored in either triplet form or
compressed column form.  Compressed column form is discussed in Section
\ref{s:intro}. Conversely, triplet form stores the contents of the matrix $A$
in three arrays \verb|i|, \verb|j|, and \verb|x| where the $k$th nonzero entry
is stored as $A ( i[k], j[k]) = x[k]$.

If the input matrix is stored in compressed column form, the functions \\
\verb|SLIP_build_sparse_csc_*| can be used. Details of these functions are
described in Sections
\ref{s:user:build_sparse_csc_double}-\ref{s:user:build_sparse_csc_mpz}.

The user should use the function that matches the data type of their available
\verb|x|. The following code snippet will show how to use these functions. Note
that this snippet serves as partially working code (i.e., select the one you'd
want to use and delete the surrounding if statements).

{\small
\begin{verbatim}
    /* Assume everything has been declared and initialized */

    /* Get the matrix A. Assume that everything is stored in
       compressed column form. This means that int* I is the
       set of row indices, int* p are the column pointers, x
       is the array of values, n is the size of the matrix and
       nz is the number of nonzeros in the matrix. We will show
       how to obtain for each possible data type of x (again,
       to have working code, select the one that fits your code
       and delete the rest)  */

    if(X IS mpz_t)
    {
        SLIP_build_sparse_csc_mpz(&A, p, I, x, n, nz);
    }
    else if (X IS double)
    {
        SLIP_build_sparse_csc_double(&A, p, I, x, n, nz);
    }
    else if (X IS int32_t)
    {
        SLIP_build_sparse_csc_int(&A, p, I, x, n, nz);
    }
    else if (X IS mpq_t)
    {
        SLIP_build_sparse_csc_mpq(&A, p, I, x, n, nz);
    }
    else if (X IS mpfr_t)
    {
        SLIP_build_sparse_csc_mpfr(&A, p, I, x, n, nz, option);
    } \end{verbatim} }

Conversely, if the input matrix is stored in triplet form, the functions \\
\verb|SLIP_build_sparse_trip_*| are used. Details of these functions are
described in Sections
\ref{s:user:build_sparse_trip_double}-\ref{s:user:build_sparse_trip_mpz}.

The user should use the function that matches the data type of their available
\verb|x|. The following code snippet will show how to use these functions. Note
that this snippet serves as partially working code (i.e., select the one you'd
want to use and delete the surrounding if statements).

{\small
    \begin{verbatim}
    /* Assume everything has been declared and initialized */

    /* Get the matrix A. Assume that everything is stored in
       compressed column form. This means that int* I is the
       set of row indices, int* J is the set of column indices,
       x is the array of values, n is the size of the matrix and
       nz is the number of nonzeros in the matrix. We will show
       how to obtain for each possible data type of x (again,
       to have working code, select the one that fits your code
       and delete the rest)  */

    if(X IS mpz_t)
    {
        SLIP_build_sparse_trip_mpz(&A, I, J, x, n, nz);
    }
    else if (X IS double)
    {
        SLIP_build_sparse_trip_double(&A, I, J, x, n, nz);
    }
    else if (X IS int32_t)
    {
        SLIP_build_sparse_trip_int(&A, I, J, x, n, nz);
    }
    else if (X IS mpq_t)
    {
        SLIP_build_sparse_trip_mpq(&A, I, J, x, n, nz);
    }
    else if (X IS mpfr_t)
    {
        SLIP_build_sparse_trip_mpfr(&A, I, J, x, n, nz, option);
    } \end{verbatim} }

Lastly, the right hand side vectors \verb|b| are populated via the
\verb|SLIP_build_dense_*| functions. Details of these functions are described
in Sections \ref{s:user:build_dense_double}-\ref{s:user:build_dense_mpfr}.

The user should use the function that matches the data type of their available
\verb|b|. The following code snippet will show how to use this function. Note
that this snippet serves as partially working code (i.e., select the one you'd
want to use and delete the surrounding if statements).

{\small
    \begin{verbatim}
    if (b2 IS mpz_t)
    {
        SLIP_build_dense_mpz(&b, b2, n, numRHS);
    }
    else if (b2 IS double)
    {
        SLIP_build_dense_double(&b, b2, n, numRHS);
    }
    else if (b2 IS int32_t)
    {
        SLIP_build_dense_int(&b, b2, n, numRHS);
    }
    else if (b2 IS mpq_t)
    {
        SLIP_build_dense_mpq(&b, b2, n, numRHS);
    }
    else if (b2 IS mpfr_t)
    {
        SLIP_build_dense_mpfr(&b, b2, n, numRHS, option);
    } \end{verbatim} }

%-------------------------------------------------------------------------------
\cprotect\subsection{SLIP LU Symbolic Analysis}
\label{s:Using:symb}
%-------------------------------------------------------------------------------

The symbolic analysis phase of SLIP LU computes the column permutation and
guesses for the number of nonzeros in $L$ and $U$. This function is called as:

{\small
    \begin{verbatim}
    /* Assume A has been populated, and option and S have been initialized. */
    SLIP_LU_analyze(&S, A, option); \end{verbatim} }

%-------------------------------------------------------------------------------
\cprotect\subsection{Simple SLIP LU Routines for Solving Linear Systems}
\label{s:Using:simple}
%-------------------------------------------------------------------------------

After initializing the necessary data structures and performing symbolic
analysis, SLIP LU obtains the solution to $A \mathbf{x} = \mathbf{b}$. Using
the ``simple'' interface of SLIP LU requires only that the user decides what data
type that he/she wants $\mathbf{x}$ to be stored as. SLIP LU allows
$\mathbf{x}$ to be returned as either \verb|double|, \verb|mpq_t|, or
\verb|mpfr_t| with an associated precision. This is done by using one of the
following functions: \verb|SLIP_solve_double| (Section
\ref{ss:SLIP_solve_double}), \verb|SLIP_solve_mpq| (Section
\ref{ss:SLIP_solve_mpq}) or \verb|SLIP_solve_mpfr| (Section
\ref{ss:SLIP_solve_mpfr}).

Below, we show sample syntax to use each of these functions. As above, this
code snippet contains all of the potential options, thus a user can merely copy
the one they desire and paste into their code.

{\small
\begin{verbatim}
    /* Assume that A, S, option, and b have been declared and populated. */
    if (USER WANTS MPQ)
    {
        // The solution is a dense matrix of size n*numRHS
        mpq_t** soln = SLIP_create_mpq_mat(n, numRHS);
        int ok = SLIP_solve_mpq(soln, A, S, b, option);
    }
    else if (USER WANTS DOUBLE)
    {
        // The solution is a dense matrix of size n*numRHS
        double** soln = SLIP_create_double_mat(n, numRHS);
        int ok = SLIP_solve_double(soln, A, S, b, option);
    }
    else if (USER WANTS MPFR)
    {
        // The solution is a dense matrix of size n*numRHS
        mpfr_t** soln = SLIP_create_mpfr_mat(n, numRHS, option);
        int ok = SLIP_solve_mpfr(soln, A, S, b, option);
    } \end{verbatim}}

On success, each of these functions return \verb|SLIP_OK| (see Section
\ref{ss:SLIP_info}).

%-------------------------------------------------------------------------------
\cprotect\subsection{Expert SLIP LU Routines}
\label{s:Using:expert}
%-------------------------------------------------------------------------------

If a user wishes to perform the SLIP LU factorization of the matrix $A$ while
capturing information about the factorization itself and solving the linear
system, extra steps must be performed that are all done internally in the
methods described in the previous subsection. Particularly, the following steps
must be performed: 1) allocate memory for $L$, $U$, the solution vector(s)
(stored as \verb|mpq_t|) $\mathbf{x}$, and others, 2) compute the factorization
$PAQ = L D U$, 3) solve the linear system $P^{-1} L D U Q^{-1} \mathbf{x} =
\mathbf{b}$, 4) permute the solution vector(s), 5) scale the solution vector if
the scaling factors of $A$ and $b$ are not zero, and 6) convert the final
solution into the user's desired form. Below, we discuss each of these steps
followed by an example of putting it all together.

%-------------------------------------------------------------------------------
\subsubsection{Allocating Memory}
%-------------------------------------------------------------------------------

Using SLIP LU in this form requires that memory be allocated for the solution
vector(s). The solution vectors are stored as an \verb|mpq_t**| array.  The
following code snippet shows how to allocate the solution vector(s).

    {\small
    \begin{verbatim}
    /* Purpose: Allocate memory for x */
    // x is of size n * numRHS
    mpq_t** x = SLIP_create_mpq_mat(n, numRHS);
    \end{verbatim}}

%-------------------------------------------------------------------------------
\subsubsection{Computing the Factorization}
%-------------------------------------------------------------------------------

The matrices \verb|L| and \verb|U|, the pivot sequence \verb|rhos|, and the row
permutation \verb|pinv| are computed via the \verb|SLIP_LU_factorize| function
(Section \ref{ss:SLIP_LU_factorize}).  Upon successful completion, this
function returns \verb|SLIP_OK|.

%-------------------------------------------------------------------------------
\subsubsection{Solving the Linear System}
%-------------------------------------------------------------------------------

After factorization, the next step is to solve the linear system and store the
solution as a set of rational number \verb|mpq_t| in the previously allocated
\verb|x| data structure. This solution is done via the \verb|SLIP_LU_solve|
function (Section \ref{ss:SLIP_LU_solve}).

Upon successful completion, this function returns \verb|SLIP_OK|.

\textbf{Note:} The solution vector given here is NOT the solution to $A
\mathbf{x} = \mathbf{b}$ because it has not been properly permuted and scaled.
Recall that when solving a system via the SLIP LU factorization, two systems
are solved: $LD \mathbf{y} = P \mathbf{b}$ and $U \mathbf{x} = \mathbf{y}$. The
solution here is the solution to $Y \mathbf{x} = \mathbf{y}$ and must still be
permuted by the column permutation $Q$ which is discussed in the next
subsection.

%-------------------------------------------------------------------------------
\subsubsection{Permuting the Solution Vectors}
%-------------------------------------------------------------------------------

Permuting the solution vector(s) is done via the function \verb|SLIP_permute_x|
(Section \ref{ss:SLIP_permute_x}).

Upon successful completion, this function returns \verb|SLIP_OK|. At the
conclusion of this routine, \verb|x| contains the solution to the scaled system
$A_{int} \mathbf{x} = \mathbf{b}_{int}$.

%-------------------------------------------------------------------------------
\subsubsection{Scaling the Solution Vectors}
%-------------------------------------------------------------------------------

Scaling the solution vector(s) is done via the function \verb|SLIP_scale_x|
(Section \ref{ss:SLIP_scale_x}).

Upon successful completion, this function returns \verb|SLIP_OK|. At the
conclusion of this routine, \verb|x| contains the solution to the system $A
\mathbf{x} = \mathbf{b}$.

%-------------------------------------------------------------------------------
\subsubsection{Converting the Solution Vector to the User's Desired Form}
%-------------------------------------------------------------------------------

Upon completion of the above routines, the solution to the linear system is
given by the \verb|mpq_t** x|. SLIP LU allows this to be converted into either
a double precision matrix or a \verb|mpfr_t| precision matrix via the functions
\verb|SLIP_get_double_soln| (Section \ref{ss:get_double_soln}) or
\verb|SLIP_get_mpfr_soln| (Section \ref{ss:get_mpfr_soln}). Below, we show how
to call these functions.

{\small
\begin{verbatim}
    if (USER WANTS DOUBLE)
    {
        double** x2 = SLIP_create_double_mat(n, numRHS);
        SLIP_get_double_soln(x2, x, n, numRHS);
    }
    else if (USER WANTS MPFR)
    {
        mpfr_t** x2 = SLIP_create_mpfr_mat(n, numRHS, option);
        SLIP_get_mpfr_soln(x2, x, n, numRHS);
    } \end{verbatim}}

%-------------------------------------------------------------------------------
\cprotect\subsection{SLIP LU Freeing all Used Memory}
\label{s:Using:free}
%-------------------------------------------------------------------------------

Upon finishing using SLIP LU all memory must be freed. As described in Sections
\ref{s:user:memmanag} and \ref{s:miscellaneous_routine}, SLIP LU provides a
number of functions to handle this for the user. Below, we briefly summarize
which memory freeing routine should be used for specific data types:

\begin{itemize}
\item \verb|SLIP_sparse*|: A \verb|SLIP_sparse* A| data structure can be freed
with a call to \verb|SLIP_delete_sparse(&A);|

\item \verb|SLIP_LU_analysis*|: A \verb|SLIP_LU_analysis* S| data structure can
be freed with a call to\\ \verb|SLIP_delete_LU_analysis(&S);|

\item \verb|SLIP_dense*|: The \verb|SLIP_dense* b| of dimension
\verb|n|-by-\verb|numRHS| can be cleared with a call to \\
\verb|SLIP_delete_dense(&b)|.

\item 2D array created via \verb|SLIP_create_*_mat|: The 2D array \verb|**x| of
dimension \verb|n * numRHS| can be cleared with a call to
\verb|SLIP_delete_*_mat(&x, n, numRHS)|.

\item 1D array of GMP data type created via \verb|SLIP_create_*_array|: The 1D
array \verb|*x| of size \verb|n| can be cleared with a call to
\verb|SLIP_delete_*_array(&x, n)|.

\item All others including \verb|SLIP_options*|: These data structures can be
freed with a call to the macro \verb|SLIP_FREE()|, for example,
\verb|SLIP_FREE(option)| for \verb|SLIP_options* option|.

\end{itemize}

\textbf{Note:} after usage of the SLIP LU routines are finished, one must call
\verb|SLIP_finalize()| (Section \ref{ss:SLIP_finalize}) to finalize usage of
the library.

%-------------------------------------------------------------------------------
\cprotect\subsection{Examples of Using SLIP LU in a C Program}
\label{s:Using:Examples}
%-------------------------------------------------------------------------------

The \verb|SLIP_LU/Demo| folder contains six sample C codes which utilize SLIP
LU. These files demonstrate the usage of SLIP LU as follows:

\begin{itemize}
\item \verb|example.c|: This example generates a random dense $50 \times 50$
matrix and a random dense $50 \times 1$ right hand side vector $\mathbf{b}$ and
solves the linear system. In this function, the \verb|SLIP_solve_double|
function is used; thus the output is given as a double matrix.

\item \verb|example2.c|: This example reads in a matrix stored in triplet
format from the ExampleMats folder. Additionally, it reads in a
right hand side vector from this folder and solves the associated linear system
via the \verb|SLIP_solve_mpq| function. Thus, the solution is given as a set of
rational numbers.

\item \verb|example3.c|: This example creates an input matrix and right hand
side vector stored as \verb|mpfr_t| numbers. Then, it shows how to create the
input matrix $A$ and right hand side vector $\mathbf{b}$ and solves the linear
system using the \verb|SLIP_solve_double| function, outputting the solution in
double precision.

\item \verb|example4.c|: This example is nearly identical to example3 except
that the input has multiple right hand side vectors and all input numbers are
stored as double precision numbers.

\item \verb|example5.c|: This example creates a random set of right hand side
vectors, reads in a matrix from a file, and solves the associated linear system
outputting the solution as a double matrix.

\item \verb|SLIPLU.c|: This example reads in a matrix and right hand side
vector from a file and solves the linear system $A \mathbf{x} = \mathbf{b}$
using the techniques discussed in Section \ref{s:Using:expert}. This file also
allows command line arguments (discussed in README.txt) and can be used to
replicate the results from \cite{lourenco2019exact}.

\end{itemize}

%-------------------------------------------------------------------------------
\cprotect\section{Using SLIP LU in MATLAB}
\label{s:Use:MATLAB}
%-------------------------------------------------------------------------------

After following the installation steps discussed in Section \ref{s:install},
using the SLIP LU factorization within MATLAB can be done via the
\verb|SLIP_LU.m| and the \verb|SLIP_get_options| functions. First, this section
will describe the \verb|SLIP_get_options| struct in Section
\ref{s:Use:MATLAB:setup} then we describe how to use the factorization in
Section \ref{s:Use:MATLAB:factor}. Again, recall that by default the SLIP LU
MATLAB routines are not natively installed into your MATLAB installation; thus
if you want to use them in a different directory please add the
\verb|SLIP_LU/MATLAB| folder to your path.

%-------------------------------------------------------------------------------
\cprotect\subsection{\verb|SLIP_get_options.m|}
\label{s:Use:MATLAB:setup}
%-------------------------------------------------------------------------------

Much like the C routines described throughout, the SLIP LU MATLAB interface has
various parameters that the user can modify to control the factorization. In
MATLAB, these are stored in a struct (hereafter referred to as the ``options"
struct) which contains 9 elements. Notice that this struct is optional for the
user to use and can be avoided if one wishes to use only default options. The
options struct can be accessed by typing the following into the MATLAB command
window:

\verb|option = SLIP_get_options;|

The elements of the options struct are as follows:

\begin{itemize}

\item \verb|option.column|: This parameter controls the column ordering used. 0
(default): COLAMD, 1: AMD, 2: no column ordering. It is usually recommended
that the user keep this at COLAMD unless they already have a good column
permutation.

\item \verb|option.pivot|: This parameter controls the pivoting scheme used.
The factorization selects a pivot element in each column as follows:

    \begin{itemize}
    \item 0: smallest pivot,
    \item 1: diagonal pivot if possible, otherwise smallest pivot,
    \item 2: first nonzero pivot in each column,
    \item 3: (default) diagonal pivot with a tolerance for the smallest pivot,
    \item 4: diagonal pivot with a tolerance for the largest pivot,
    \item 5: largest pivot.
    \end{itemize}

It is recommended that the user always selects either 3 or 1 for this parameter
UNLESS they are trying to extract the Doolittle factors, then 5 may be
appropriate (due to the size of numbers in Doolittle).

\item \verb|option.int|: Set this parameter equal to 1 if the input matrix is
already integral. Otherwise, if the input matrix has any decimal entries,
scaling must be performed to obtain an integral input matrix. \import If the
input matrix is not integral and this parameter is set equal to 1, the values
will be truncated.

\item \verb|option.intb|: Set this parameter equal to 1 if the input right hand
side vector(s) are already integral. Like the input matrix, if $\mathbf{b}$
contains any fractional entries, scaling must be performed to ensure
integrality.

\item \verb|option.tol|: This parameter determines the tolerance used if one of
the threshold pivoting schemes is chosen. The default value is 0.1 and this
parameter can take any value in the range (0,1).

\end{itemize}

%-------------------------------------------------------------------------------
\cprotect\subsection{\verb|SLIP_LU.m|}
\label{s:Use:MATLAB:factor}
%-------------------------------------------------------------------------------

The \verb|SLIP_LU.m| function solves the linear system $A \mathbf{x} =
\mathbf{b}$ where $A \in \mathtt{R}^{n \times n}$, $\mathbf{x} \in
\mathtt{R}^{n \times m}$ and $\mathbf{b} \in \mathtt{R}^{n \times m}$. The
final solution vector(s) obtained via this function are exact prior to their
conversion to double precision.

The SLIP LU function expects as input a sparse matrix $A$ and dense set of
right hand side vectors $\mathbf{b}$. Optionally, the user can also pass in the
options struct. Currently, there are 2 ways to use this function outlined
below:

\begin{itemize}

\item \verb|x = SLIP_LU(A,b)| returns the solution to $A \mathbf{x} =
\mathbf{b}$ using default settings. The solution vectors are more accurate than
the solution obtained via \verb|x = A \ b|.

\item \verb|x = SLIP_LU(A,b,option)| returns the solution to $A \mathbf{x} =
\mathbf{b}$ using user specified settings from the options struct.

\end{itemize}

%-------------------------------------------------------------------------------
% References
%-------------------------------------------------------------------------------

\newpage
\bibliographystyle{siam}
\bibliography{SLIP_LU_UserGuide.bib}
\end{document}

